\begin{abstract}
Eldrebølgen er over oss, og vi forventer betydelig økte kostnader i
helsesektoren som et resultat av en befolkning som stadig lever lengre.

\textit{Velferdsteknologi} er en rekke løsninger som skal hjelpe mennesker til
å ha høyere livskvalitet, være mer selvstendige og bo lengre hjemme. Ett av
disse er trygghetsalarmen.

Imidlertid har dagens alarmer betydelige mangler.  Halvparten kan ikke brukes
ute eller til lokalisering.  Kun én har skjerm, men blir ikke brukt til
interaksjon. Ingen kan oppdateres med nye funksjoner etter forsendelse.
Samtidig sliter hjemmesykepleien med effektiv vaktorganisering, og det fører
til økte kostnader.

Vår løsning er en avansert trygghetsalarm med funksjonalitet på høyde med
dagens smarttelefoner. Fordi vi bruker en tynnklientløsning kan den produseres
med billige komponenter og aktivere nye funksjoner ved kjøp.  Ved å utstyre
hjemmesykepleiere med samme enhet kan et datasenter koble alarmanrop til den
nærmeste som kan hjelpe.  Flere menyvalg for brukeren kan hjelpe med
prioritering.  Sammen bidrar dette til effektivisering og reduksjon av
kostnader og responstid.

Vi har et investeringsbehov på omtrent 4,5 millioner kroner, og forventer å
tjene dette inn ved utgangen av det femte året.  Årsoverskuddet for det femte
året er beregnet til 7 millioner kroner med en markedsandel på kun 10\%.
\end{abstract}
