\begin{abstract}
Eldrebølgen er over oss, og vi forventer betydelig økte kostnader i
helsesektoren som et resultat av en befolkning som stadig lever lengre.

\textit{Velferdsteknologi} er en rekke løsninger som skal hjelpe mennesker til
å ha høyere livskvalitet, kunne mer selvstendige og bo lengre hjemme. Ett av
disse er trygghetsalarmen.

Imidlertid er mangler dagens trygghetsalarmer betydelig funksjonalitet.
Halvparten kan kun brukes hjemme og har ikke sporing. Kun én har skjerm, og
ingen kan oppdateres med nye funksjoner etter forsendelse.  Samtidig sliter
hjemmesykepleiere med effektiv organisering, som videre øker kostnadene
involvert i hjemmehjelpen.

Vår løsning er en avansert trygghetsalarm med funksjonalitet på høyde med
dagens smarttelefoner. Den er billig å produsere fordi vi bruker en
tynnklientløsning. Ved å utstyre hjemmesykepleiere med samme enhet kan et
datasenter koble alarmoppringninger med den nærmeste som kan hjelpe. Dette
bidrar til effektivisering. Flere menyvalg for brukeren kan hjelpe med
prioritering.

Vi har et investeringsbehov på omtrent 4,5 millioner kroner, og forventer å
tjene dette inn ved utgangen av det femte året.  Årsoverskuddet for det femte
året er beregnet til 7 millioner kroner med en markedsandel på vel 10\%.
\end{abstract}
