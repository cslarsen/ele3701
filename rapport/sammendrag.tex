\begin{abstract}
Eldrebølgen er over oss, og vi forventer betydelig økte kostnader i
helsesektoren som et resultat av en befolkning som stadig lever lengre.

\textit{Velferdsteknologi} er en rekke løsninger som skal hjelpe mennesker til
å ha høyere livskvalitet, være mer selvstendige og bo lengre hjemme. Ett av
disse er trygghetsalarmen.

Imidlertid har dagens alarmer betydelige mangler.  Halvparten kan ikke brukes
ute eller til lokalisering.  Kun én har skjerm, men bruker en plastknapp som
primærinteraksjon.  Ingen kan oppdateres med nye funksjoner etter forsendelse.
Samtidig sliter hjemmesykepleien med effektiv vaktorganisering, og det fører
til økte kostnader.

Vår løsning på disse problemene er en avansert trygghetsalarm på høyde med
dagens smarttelefoner.  Fordi vi bruker en tynnklientløsning kan den produseres
med billige komponenter, og nye funksjoner kan kjøpes og aktiveres momentant.

Ved også å selge dem til hjemmesykepleiere kan et datasenter koble alarmanrop
til nærmeste person.  Brukervennlige menyvalg og talestyring gjør at brukeren
kan velge alarmkategori, slik at anropet prioriteres automatisk.  Sammen bidrar
dette til reduksjon av helsekostnader og kortere responstid i akuttmedisinske
tilfeller.

Vi har et investeringsbehov på omtrent 4,5 millioner kroner, og forventer å
tjene dette inn ved utgangen av det femte året.  Årsoverskuddet på dette
tidspunktet er beregnet til 7 millioner kroner med en markedsandel på knappe 10\%.
\end{abstract}
