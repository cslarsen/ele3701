\begin{abstract}
Dagens smarttelefoner kjører applikasjoner på selve enheten.
De består derfor av kraftige komponenter som er batterikrevende og dyre.
% Ta med BEHOV for konsumentene.

Samtidig har vi sett at maskinvare i dag har blit virtualisert, slik at de kan
flyttes over kloden gjennom nettet, mens de kjører. Stasjonære PC-er kan brukes
gjennom tynne klientgrensesnitt, og spillkonsoller kan leies over internett med
ekstremt enkle bokser som kobles til TV-en.

Vår visjon er å introdusere disse konseptene til mobile enheter ved å utvikle
en tynnklient mobilplattform, bestående av maskin- og programvare.
%
Kompleksiteten flyttes da fra mobilen over til tjenere i nettskyen. Dette leder
til billigere, sikrere enheter med nye bruksmuligheter.

For å realisere denne visjonen vil vi i første omgang fokusere på å lage en
trygghetsalarm som kjører på denne tynnklientplattformen. Den skal selges som
et eget produkt med en fast pris og et månedsabonnement.
\end{abstract}
