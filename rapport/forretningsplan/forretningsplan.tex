\chapter{Forretningsplan}
\label{del1}

% Denne bør egentlig være executive summary.
\section{Idé og visjon}

I prosjektskissen presenterte vi en visjon om å lage en \textem{tynnklient}
smarttelefon. Telefonen vil da bli en dum terminal som formidler interaksjon
til servere som kjører applikasjoner i skyen. Skjermbilder sendes ned til
telefonen.

Argumentet var at det ville lede til lavere produksjonskost, lengre
batteritid, høyere brukersikkerhet, samt at telefonhuset ville bli lettere og
mindre.  Imidlertid vil dette lede til høyere datatrafikk, og våre estimater
viser at det vil kreve et gjennomsnittlig månedlig forbruk på rundt 10 Gb.

Dette er fremdeles vår visjon, men tiden er ikke enda moden for slike
datamengder. På sikt vil sannsynligvis slike datamengder bli vanlige og dermed
billigere, men akkurat nå er dette for tidlig.

Istedenfor har vi gjort en såkalt \textem{pivot} --- en endring av
konseptet --- til å produsere \textem{tynnklient smartklokker}.  På grunn av
mindre skjermer vil man kunne tjene penger på en slik modell, og man får samme
fordeler. 

Smartklokker er et nytt marked hvor både store aktører som Apple og Samsung
satser tungt. Man har også mindre, innovative produsenter som Pebble, verdt omtrent \${}1.4-1.8 mrd i dag, m

Med introduksjon av smartklokker fra produsenter som Apple, Samsung, Pebble og
flere kan markedet vokse fra \${}1.4-1.8 milliarder idag til \${}10
milliarder i 2018 \cite{citi.grow}.

Dette er også et nytt marked med store aktører som Apple og Samsung
og mindre innovatører som Pebble og Tinitell. Man anslår at markedet vil vokse
fra NOK 10-14 milliarder i dag til 76 milliarder i 2018 \cite{citi.grow}.



% Må begynne med å snakke om markedet og om behov. Altså snakke om vår mulighet
% for å gjøre noe med et problem og tjene penger på det. Snakk om tall, snakk
% om trender, snakk om folk, hva store aktører gjør og spå litt i fremtiden.
% Forklar hvorfor vårt valg av forretning er lurt.

\section{Forretningsplan}

% Forretningsmål må være tydelig definert, sånn man vet hva man faktisk ønsker
% å gjøre. Akkurat nå er det litt sprikende. Visjonen vår er på sikt tynnklient
% smarttelefon og tablets, men vi må begynne smått: Derfor satser vi på klokke,
% gjerne vinklet med spesifikke nisjer som trygghetsalarm for å få traction.
% Deretter skal vi vokse oss ut til vanlige forbrukere. Dette kan gjøres
% samtidig, pga løsningen vår er så generell og lett kan tilpasses på en måte
% som er umulig for Apple/Samsung. (De leverer ett produkt som kan gjøre alt;
% vårt produkt KAN styres 100% fra servere, derfor kan det spesialtilpasset
% slik det kun har ett formål; altså kan kun være trygghetsalarm, og ingenting
% annet.)

% Viktig å få med at vi ønsker abonnementsavtale. Det har ikke Apple/Samsung.
% De selger device primært og deretter apps. Jmf/ref siste earnings rapport fra
% Apple, de har grafer på hva de tjener peng på. De tjener mest på salg av
% device, og deretter ca 12% (??) på apps. De har _ingen_ månedlige inntekter,
% det har vi. Hvis vi regner med en klokke varer 3 år (sikkert lengre enn en
% smarttelefon) så kan vi sammenligne totalinntekten over 3 år vs Apple sin
% inntekt per telefon (de lever også ca 3 år; back dette opp). Så de selger
% telefon og tjenester for X kroner over 3 år, vi tjener Y kroner på device +
% tjeneste over 3 år, men har mye mer jevn og forutsigbart salg.

% Det jeg skrev over er hele forretningsplanen: Tynnklient + abonnement,
% punktum. Sekundære mål: Spesialapps (1881, Trygghetsalarm, osv). 

% Tertiære mål, nesten exit-strategier: OEM, selge komponenter, selge trådløse OEM komponenter for å styre
% vending machines, kaffemaskiner, terminaler for SAS osv. Hvorfor? Fordi dette
% er ting som ALLTID _spesiallages_ til hvert enkelt formål. Vi bygger jo opp
% en FELLES plattform som vil funke på ALLE slike devices, fordi skjermbildet
% skjer i skyen. Blir sprikende mål, derfor må de deles inn: Det viktigste er
% plattformen, og i første omgang å få inn jevnt med penger sånn vi kan prøve
% oss på andre typer produkter. ALLE produktene vil styrke plattformen.
% Apple/Samsung gjør IKKE dette, og ignorerer dette helt (kan være av en
% grunn). De PRØVER feks med iPad på fly, men de får det ikke helt til.

% Exit-strategier: Bli kjøpt opp, pivot til OEM. Beskyttelse ved å bygge opp IP
% portefølge med patenter osv. Andre fordeler med vår plattform: Kan kode i hva
% språk en vil.

% NDE: Selge spesialløsning for trygghetsalarm, vending machine. Bygge opp
% beskyttelse i form av IP, patenter -> dette kan selges i helhet eller som
% OEM. Visjon er å lage mobiltelefon. Men mål nr 1 er å lage en klokke som
% selges til kunder, vi MÅ tjene penger tidlig ellers går alt i vasken. Vi
% kunne startet med OEM, men da renner tiden ut. Den er nå gylden. Er nesten
% egentlig litt seint ute, men se på Pebble, de har vært en stund og ser ut som
% Apple kommer til å _knuse_ de. Hvorfor? Ekkelt møkkaprodukt uten noen form
% for hooks. Vår hook er jo tynne devices med lav produksjonskost og enorm
% batteritid. Vi kommer til å vinne. Så bygger vi opp plattform. Mobil
% tynnklient krever for mye båndbredde, men ikke telefon. Og tablets er det
% ikke viktig, men de er såpass avanserte at det er vanskelig å komme inn på
% markedet <=== VIKTIG POENG, ergo derfor satser vi på klokke strategisk.

% TODO: Reinskriv alt dette over.

% Forretningsmål

Målet er å tjene penger ved å selge produktet.
Vi bør ha ca 500-1000 kunder for å nå $10-20m$ i omsetning.
Må ha nok penger til å ha en del ansatte, minst 10 pluss support, selv om vi
satser på self-service.

Starte å selge i Norge, men vi tenker globalt fra første dag.
Vi skjønner kulturen i dette markedet. Kunne satset på Tyskland, med 90
millioner innbyggere (som er et av primærlandene til Apple).

Kan eventuelt selge til hvem som helst over nettet, men må deles i regioner,
også på infrastruktur.

Andre åpenbare mål som må tas med?

Et godt forretningsmål er jo å tjene penger på å tilby noe som folk trenger.
Mens Apple sitt fokus er på helse og kommunikasjon, er vårt fokus på trygghet,
for eksempel. For gamle, for barn, osv. (Se på den klokka som Stian fant).

Exit-strategier er NDE, OEM, IP, patenter, komponenter, selge hele eller deler
av dette.

% Grunner til at målene kan oppnås

Nytt voksende marked. Når de store satser kan vi være med i dragsuget. Er også
nytt marked, så folk har ikke dannet en oppfatning av hva denne typen
produktkategorien kan tilby. Altså hva de forventer av et slikt produkt. Dermed
kan vi være med å forme denne oppfatningen, men da må vi være kjappe. Dette er
en fordel. Kan være med og definere spillereglene.

MÅ HA FLERE GRUNNER!

Vår løsning er tynnklient, betyr at mye av kompleksiteten flyttes over til
serverne, som er fordel for produksjon (lav kostnad, enkelt å utvikle) av
klokke og apps: Lang batteritid, billigere å produsere pga mindre komponenter,
tynnere av samme grunn, evt bruke plassen til batteri. ENkelt å programmere
for, fordi man trenger ikke lage et program som skal tvinges inn i en
bitteliten dings som nesten ikke har strøm.

En edge med batteri og enkelhet.

Virtualisering: Lett å ekspandere med cloud-løsninger (CDN).

% MERK: Hvis disse grunnene skal funke må det være veldig bra forklart hva
% produktet er. Mulig skissen funker som dette, men det bør gjentas i oppgaven
% her, kort.

Relativt lav inngangsbillett pga produktet er ikke så komplisert som feks en
smarttelefon.

Tilbyr noe unikt! PARTNERE med trygghetsalarm, brannvesen, securitas, danne
partnerskap.

MÅ HA MER.

På siden:
Fred.Olsen har allerede via sitt Timex firma laget en klokke som alltid er på
nett hele tiden i hele USA. Ergo er teleoperatører villige til å være med på
sånt. Vi MÅ ha en sak som funker i utlandet, der vi har partnere/avtaler men
også tar kosten for de få kundene som er ute (altså vi spanderer det, usynlig
for kunden).

% Planer for å nå målene

Hva er målene? Har jo skissert noe over, må bare sorteres i riktig rekkefølge.

Prototype, teste, markedsføring, crowdsourcing? kanskje ikke i begynnelsen, pga
vi vil være stealth. Og crowdsourcing må ha et vanvittig klart
konsument-appeal.

% Hva slags type team trenger vi?
Teamet vi trenge: Utviklere primært, og spesielt i begynnelsen. Software- og
hardware-utviklere. Alt faller på at prototypen funker i praksis, dvs at den
når mål om strøm og pris på abonnement. Når dette er iorden må vi danne
partnerskap. Vi trenger dermed fort businesspersoner, etterhvert må vi etablere
infrastruktur for salg og support osv.

Fordel: Vår klokke kan spesialtilpasses LETT ulike mål. Istedenfor generell
smartklokke kan vi LETT differensiere produkt med samme device, men annen
software i skyen, som trygghetsalarm. Dritbra poeng for å få opp salg, mens
dette går og tjener peng blir IP og plattform utviklet. Men det drar også fokus
vekk fra vårt mål: Massekonsumentene (altså vanlige folk), og tar alltid mer
tid og ressurser enn forventet. Men det kan være en smart måte å begynne på og
få traction. MEN vi har dårlig tid, smartklokkene fra de store er allerede her.
Prototypen kunne dermed vært å lage en trygghetsalarm, men er som sagt redd for
det tar for mye fokus vekk. Når vi tjener penger på denne typen produkt så blir
det vanskelig å endre organisasjonen til noe nytt. Altså, vi kan ikke gjøre
begge deler samtidig, men én av tingene først.

Alarm: Voldektsalarm -> GPS -> politi. Fordel kan glemme mobilen hjemme, klokka
funker likevel. Megafordel ifht Apple Watch.

% Basisstruktur

Markedsanalyse:

Relatert industri: Software, alarm, klokke, smarttelefon, IT, vending machines
osv.

Markedsbevissthet: STOR, alle venter på det. Hvor fort kan vi imidlertid kome
igang? (Dette må være med i SWOT, "window of opportunity" minker fort, vi er
ikke first mover så har ikke den fordel. MEN vi lager ekte INNOVASJON, og kan
være lurt som små å følge de store (tror dette står i boka)).

MVP (minimum viable product) = Trygghetsalarm, hvis vi får laget avtale. TODO:
Slå opp om trygghetsalarm og se hvordan de funker. Tror de må ha hussentral,
men hvor ofte må de lades? Evt kan man lage større boks med batteri i.
Må bestemme om strategi med å lage trygghetsalarm er den lure måten å gå på.
Microsoft begynte med basic compilere på alle maskiner, og så OS. To
forskjellige ting, men samtidig så utrolig logisk progresjon; de solgte
allerede til alle. Vi må ikke være dumme og prøve oss på to typer produkt.
Men sånn sett så ER det samme produkt: Nøyaktig samme device, bare annen
software. Så det funker. Men kan være vanskelig å komme inn på markedet for
trygghetsalarm, fordi det er allerede mange om beinet. Og vår løsning er gjerne
dyrere (eller? Sjekk priser!!! Kanskje vår er INSANELY mye billigere, fordi man
ikke trenger hjemmesentral? Pluss vi kan faktisk sjekke om devicen funker,
remotely! Og kan ha human operatører. Og device kan brukes til noe mer.) Denne
strategien må diskuteres med noen.

Kunnskap, forskning, resultat? Finn noe og ref det.

Størrelse på marked: Ca 10 mrd, antatt, vokse til ca 100 mrd. Har artikkel,
må lese den og se hvordan de har argumentert.

VIKTIG OG DRITBRA: 1881, de bommer TOTALT på appene sine. De tilbyr kun oppslag
i telefonkatalog, som er veldig enkelt å kopiere. Så derfor har de massiv
markedsføring. De har også en telefontejeneste hvor en kan spørre om hva en
vil. Hvorfor er ikke denne tilgjengelig fra mobil? Jo, den er, de må ringe, men
vi har en logo med 1881, klikk, snakk med person, regning på abb. Kan bli en
partner. Tenk "sekretær-app" a la magic.

Kritiske behov for konsumenter? Veeel. Det er jo en klokke, så behovet er å
vise tiden. Det kan hvem som helst gjøre. Apple gjør det mer fancy bare, det er
alt de gjør. De satser på helse. Vi kan satse på ting som trygghet.

Eldre: Trygget + alarm
Barn: Trygget + kommunikasjon med foreldre
Andre: Nyttig dings som viser tiden, m.m. Gjør hverdagen enklere.

Målmarked: Norge, Europa. Størrelse på dette da? Tjaaa... Tror at folk kommer
til å se fordelen med smartklokke. Man betaler jo lett mange mange tusen for en
fancy klokke, men må likevel stille den av og til. De dyreste har
månedsvisning, dvs hvilken dag det er. Denne må stilles. Er helt ånnas,
spesielt når en vanlig digitalklokke er bedre sånn sett. Så når folk ser hvor
bra det er med smartklokke så vil nok markedet vokse. DOG, fancyklokker er jo
motedings, men jeg tror markedet for vanlige klokker minsker inn. OEM forøvrig,
mange som gjør det i Sveits, BOSS-klokker og alle egentlig er produsert av ett
stort firma. Kan jo sjekke hva de tjener. De lager urhuset, så lager boss
designet. Så her snakker vi om penger faktisk! Kanskje dette er lurere? Med
tynnløsning kan de jo lage den så spesifikk de vil.

Trenger REFS og TALL. Bruk biblioteket. Lag diagrammer for å vise hvordan alt
henger sammen. Lag grafer også, det er alltids gudd.

Demografi: Barn, alle, eldre. Land? Vestlige primært. Klarer vi å lage billig
nok kan vi satse i India og slike land.

Vår hovedprofil: Altså en nyttig smartklokke, beveger oss mot personlig
menneskelig kontakt... tjaa 

Apple: Utviklere kommer til dem. Oss: Vi må gå til utviklere, dvs partnere.
Dette er en strategi / plan.

Vi trenge altså en identitet.

Partnere: Telenor, 1881, hjemmesykepleien / trygghetsalarm, politi, sikkert
uniteressant for de. Brannvesen? Alarmselskaper? Har ikke homekit, men er lite
homekit-stuff på markedet likevel. Kan ikke lade bil via klokka dog.

HMM, tenk på det da. Vi har egentlig ikke sjanse å konkurrere med Apple og
Samsung. De SELGER klokka NÅ. Så vi må satse annerledes taktisk, så er
strategien å konkurrere med dem likevel. Ny type alarmtjeneste: Kan brukes hvor
som helst, får hjelp der og da. Viking, for å hente bilen (jaja, du kan jo bare
ringe egentlig).

Demografi, lokasjon: Norge, når VI skal gå globalt så henter vi investorpenger.
Utenlandske er alltid imponert over suksess i hjemlandet (jmf Karl Ove
Knausgård med NYT og Röyksopp). Kan da si at TIL TROSS for det er lite folk i
Norge så gjør vi det bra.

Sesonsvariasjon: Ja, når folk reiser på ferie så er det kjipt om klokka ikke
funker. Beste er om kunden ikke betaler ekstra for dette, så tar vi kosten i
disse månedene, basert på inntekt fra de andre.  Kan si klokka kun funker i
Europa og sånt dog, altså EU + USA. Evt betale mer for andre regioner. Men
samme problem har Apple med EU. Kun i USA det er store teleoperatører.  Kan si
klokka kun funker i Europa og sånt dog, altså EU + USA. Evt betale mer for
andre regioner. Men samme problem har Apple med EU. Kun i USA det er store
teleoperatører.

Prising: TALL. 1000 + 100? Eller var det 200? Tenk på tvillingsim. Kan vi
progge simkort i rewritable firmware? Tja, de bruker jo antenner og sånt
allerede fra Ericsson og Nokia, så det spørs. Finn ut hvem som lager simkort
da. Bør nok satse på tvillingsim, men da må vi ha eget abbavtale over dette.
Dvs, da kan vi også egentlig drite i dataforbruket, men da blir kunder sure
hvis det bruker mye. Så vi MÅ ha avtale. Av type: Noen bruker mye, de fleste
bruker mindre enn avtalen, ergo går det i overskudd totalt sett.

Orgstruktur? Colo-labs, vis til boka. TODO: Blir så lige i begynnelsen av det
ikke har noe å si.

SWOT: TODO. Finn pris på 1881.

Distribusjonskanaler: Nett (salg på nett), hjemmehjelp, partnere. Flest oom vi
ikke trnger eget simkort.

Kunderelasjon: TODO, slå opp og finn ut hva som ligger i dette.

Nøkkelressurser og kritiske suksessfaktorer: Prototype. At datatrafikken
faktisk blir som vi har beregnet.

Parnersalg, partnere som telenor? Ah, ja, det var jo det. Vi ville ha
tvillingsim slik at klokka ringer samtidig med telefon. Men det trenger vi
egentlig ikke om vi har direkte avtale med Telenor. Dvs da blir dette et eget
abb utenom telefonen. Er folk villige til dette da?

Kjerneaktivitet: Finn ut hva dette handler om.


\section{SWOT-analyse}

\subsection{Styrker}
Internt, eksternt
\begin{itemize}
  \item Billigere mobilenheter
  \item Lengre batteritid (kanskje!)
  \item Tettere kundeforhold
  \item Mer dynamisk (CPU, minne, disklagring)
  \item Bedre sikkerhet (fra stjeling av mobil)
\end{itemize}

\subsection{Svakheter}
Internt, eksternt
\begin{itemize}
  \item Energibruk radio er ukjent
  \item Dårlig mobildekning, må ha wifi da
  \item Latency
  \item Kostnad ifbm server
  \item Hvis serverne er nede er mobilen ubrukelig
  \item Hvorfor skal kundene velge oss? Vi må vite nøyaktig hva vi frir til.
\end{itemize}

\subsection{Muligheter}
\begin{itemize}
  \item Kulere, tynnere mobil
  \item Konkurrenter prøver å få til det vi begynner med (tynne mobiler)
  \item De sliter med cloudløsninger fordi de hele tiden må velge
  hybridløsninger (vi har alt på cloud fra begynnelsen av)
\end{itemize}

\subsection{Trusler}
\begin{itemize}
  \item Lett for de store å imitere, men de mangler da markedsprofil (feks
      Samsung må lage ny mobil men må markedsføre under Samsung-navnet, vi har
      eget navn som er ensbetydet med konseptet)
  \item De kan lett kopiere, de har allerede mange bra hybridløsninger. De kan
  flytte mer over til cloud, men har stor kundemasse allerede. Vi kan vokse med
  kundemassen og får nok en first-mover advantage.
\end{itemize}

% Dette her er ganske tynt foreløpig! Bare få inn alt her, så kan det
% finjusteres etterhvert.

\section{Kundesegment}


\section{Kundeverdi og verdiløfte}

Alt ligger i skyen: Telefonen er bare et vindu inn til dine apps.
Sikkerhet. Oppgradering skjer i stor grad på serversiden (mer lagringsplass? du
betaler for det du bruker, automatisk.. får regning, så enkelt er det. Ikke
vits å kjøpe mer plass, du bare betaler for det du bruker, dvs pris per gig per
måned. Mer CPU-kraft? Dette er også rullende, alt etter hvor mye du bruker. Så
hvis du kjører mandelbrot-program så koster det mer.

Komprimering mellom enhet og sky: All data kan i prinsippet komprimeres, og alt
streames. Dvs når du hører musikk streames den, stopper du den så stopper
streamen. Trenger ikke lagre ting lokalt. PÅ serversiden er en filkopiering
bare en refcount som økes. Dette gjelder også bilder og sånt, feks hvis man
deler et bilde til mange venner så er det bare snakk om å refcounte det ene
bildet.

Support, kan dele skjerm med supportfolk, så de kan hjelpe deg.

Husk at folk betaler ofte for ting som dropbox. Vår løsning har jo dette
integrert, for mobil-only altås, men sammenlignet med iDrive så er det del av
greia... må jo ha en del lagring gratis fra start.

\section{Distribusjonskanaler}

\section{Kunderelasjon}

\section{Inntektsstrøm}

\section{Nøkkelressurser og kritiske suksessfaktorer}


\section{Kjerneaktiviteter}

Salg, infrastruktur, utvikle apps, tilby utviklerplatform.

\section{Partnere}


\section{Kostnadsstruktur}

Skille mellom kostnad for å få opp alt og selge første telefon, deretter
operativ drift etter vi har kunder. Blir en overgang, hvor det brennes mye
penger i begynnelsen og så stabiliserer seg over tid. Over tid vil man også se
etter måter å redusere kostnader på, feks inngå volumavtaler, flytte apps
nærmere kunden (CDN) osv. Disse må være TALLFESTET.

Datasenter: Hvor mange brukere per maskin? Husk at vi i prinsippet tillater kun
1 app å kjøre til enhver tid. De som ikke er aktive saver state (i API, bør
være billig som i kreve lite lagringsplass).

Drift av sentre, båndbredde, voip, sms-gw, osv. Blir heavy på denne siden, a la
altibox. Må ha CDN osv.

R\&{}D blir heavy i begynnelsen, faktisk en syk kost for å få det til.

Telefon: Må beregne å ta mellom 2-3 tusen for den? Regna litt på det.

Må også se på inntekter... husk at datatrafikk blir en del, men vi MÅ over på
wifi. Ser for meg 2500 for telefon og hvis vi ikke har integrert dataabb så må
det koste 150 i mnd, dette på toppen av dataabb som blir fort 300. Så blir jo
egentlig ganske dyrt. Vi kan lure inn en avtale som egen operatør, men det blir
fremdeles dyrt. Target må være 300 kr per måned som tak, ellers vil jo folk
velge andre tjenester. Poenget er jo det å ha en dritbillig tlf.

ALTERNATIV: Lage en smartklokke som er tynnklient? Dritvanskelig å lage så små
devices, men den kan være ypperlig siden det er mindre skjerm = mindre data,
den trenger virkelig dette med å kunne offsette cpu-kraft til skyen. Dette er
også et marked som ikke fins enda, alle prøver seg, så det må være "lov" å
eksperimentere her.. fins jo en klokke (fred.olsen) som er always on (AT\&{}T),
får sikkert med telenor på dette.

Bruke mirasol display... e-ink, bruker lite strøm. Trenger ingen backlight. Kan
ses i sollys.

Pluss, folk er vel egentlig ikke interessert i å se video på små klokker, men
vil da heller bruek den til andre ting.. og vi kan tilby masse gudd shitt,
spesielt siden man ikke trenger mye lagringsplass... inkl sende og motta
beskjeder (via diktering, som skjer serverside), samtaler (enkelt å rute)..
tror ikke apple watch har noe mobilgreier i seg, så her har vi fordel.. MÅ være
integrert sim, så har et lite window of opportunity her.

Se på value proposation whitepaper for mirasol, står en del interessante tall
der som vi kan bruke.

Hvem vil ha denne da? Faktisk kan vi til og med selge til de som ikke vanligvis
bruker mobil. Gamle folk kan ha et kodeord de sier, så aktiveres en
hjelpefunksjon (feks operatør) som kan hjelpe dem. Kan brukes av klatrere,
redningsfolk.. vanlige folk som er på farten, osv. Eneste er integrasjon m
mobil som blir et problem, vi bygger jo egen silo her.

"Eventually we're going to make a mobile phone", sånn pitch-aktig.

USA er egentlig perfekt marked, for vi kan ha avtale på hele USA. I Europa er
det mer stress, men det har blitt gjort for enkle ting (feks TomTom).

Apple watch resolution: 340px height liten modell, 390px stor modell.
Så ca 270x340 vs 312x390
Det blir 270*340=92k vs 312*390=121k
Hvor mye kan PNG komprimere? GIF har typisk 4:1-10:1, PNG 10-30 prosent mindre,
En kombinasjon av JPEG+lossless funker og blir mye brukt... JPG har jo i medium
30-50:1, så vi kan regne med kanskje 1:10-1:20? da snakker vi om 12k-24k for et
bilde. Animasjoner: Kan ha swipe-animasjoner som default, sånn internt minne
bufrer 4 skjermer (for swiping i alle retninger), swiping er da smooth.
Fabrice bellard sin sak, BPG, gir vel en bra komprimering vs kvalitet?
Så hiver på komprimering, tipper vi kan nå 1/10, så 9-12k per bilde da.

Tror dette kan funke...

