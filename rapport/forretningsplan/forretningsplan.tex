\chapter{Forretningsplan}
\label{del1}

\section{Idé og visjon}

I prosjektskissen presenterte vi en visjon om å lage en \textem{tynnklient}
smarttelefon. Telefonen vil da bli en dum terminal som formidler interaksjon
til servere som kjører applikasjoner i skyen. Skjermbilder sendes ned til
telefonen.

Argumentet var at det ville lede til lavere produksjonskost, lengre
batteritid, høyere brukersikkerhet, samt at telefonhuset ville bli lettere og
mindre.  Imidlertid vil dette lede til høyere datatrafikk, og våre estimater
viser at det vil kreve et gjennomsnittlig månedlig forbruk på rundt 10 Gb.

Dette er fremdeles vår visjon, men tiden er ikke enda moden for slike
datamengder. På sikt vil sannsynligvis slike datamengder bli vanlige og dermed
billigere, men akkurat nå er dette for tidlig.

Istedenfor har vi gjort en såkalt \textem{pivot} --- en endring av
konseptet --- til å produsere \textem{tynnklient smartklokker}.  På grunn av
mindre skjermer vil man kunne tjene penger på en slik modell, og man får samme
fordeler. 

Smartklokker er et nytt marked hvor både store aktører som Apple og Samsung
satser tungt. Man har også mindre, innovative produsenter som Pebble, verdt omtrent \${}1.4-1.8 mrd i dag, m

Med introduksjon av smartklokker fra produsenter som Apple, Samsung, Pebble og
flere kan markedet vokse fra \${}1.4-1.8 milliarder idag til \${}10
milliarder i 2018 \cite{citi.grow}.

Dette er også et nytt marked med store aktører som Apple og Samsung
og mindre innovatører som Pebble og Tinitell. Man anslår at markedet vil vokse
fra NOK 10-14 milliarder i dag til 76 milliarder i 2018 \cite{citi.grow}.



% Må begynne med å snakke om markedet og om behov. Altså snakke om vår mulighet
% for å gjøre noe med et problem og tjene penger på det. Snakk om tall, snakk
% om trender, snakk om folk, hva store aktører gjør og spå litt i fremtiden.
% Forklar hvorfor vårt valg av forretning er lurt.

\section{Forretningsplan}

% Forretningsmål må være tydelig definert, sånn man vet hva man faktisk ønsker
% å gjøre. Akkurat nå er det litt sprikende. Visjonen vår er på sikt tynnklient
% smarttelefon og tablets, men vi må begynne smått: Derfor satser vi på klokke,
% gjerne vinklet med spesifikke nisjer som trygghetsalarm for å få traction.
% Deretter skal vi vokse oss ut til vanlige forbrukere. Dette kan gjøres
% samtidig, pga løsningen vår er så generell og lett kan tilpasses på en måte
% som er umulig for Apple/Samsung. (De leverer ett produkt som kan gjøre alt;
% vårt produkt KAN styres 100% fra servere, derfor kan det spesialtilpasset
% slik det kun har ett formål; altså kan kun være trygghetsalarm, og ingenting
% annet.)

% Viktig å få med at vi ønsker abonnementsavtale. Det har ikke Apple/Samsung.
% De selger device primært og deretter apps. Jmf/ref siste earnings rapport fra
% Apple, de har grafer på hva de tjener peng på. De tjener mest på salg av
% device, og deretter ca 12% (??) på apps. De har _ingen_ månedlige inntekter,
% det har vi. Hvis vi regner med en klokke varer 3 år (sikkert lengre enn en
% smarttelefon) så kan vi sammenligne totalinntekten over 3 år vs Apple sin
% inntekt per telefon (de lever også ca 3 år; back dette opp). Så de selger
% telefon og tjenester for X kroner over 3 år, vi tjener Y kroner på device +
% tjeneste over 3 år, men har mye mer jevn og forutsigbart salg.

% Det jeg skrev over er hele forretningsplanen: Tynnklient + abonnement,
% punktum. Sekundære mål: Spesialapps (1881, Trygghetsalarm, osv). Tertiære
% mål: OEM, selge komponenter, selge trådløse OEM komponenter for å styre
% vending machines, kaffemaskiner, terminaler for SAS osv. Hvorfor? Fordi dette
% er ting som ALLTID _spesiallages_ til hvert enkelt formål. Vi bygger jo opp
% en FELLES plattform som vil funke på ALLE slike devices, fordi skjermbildet
% skjer i skyen. Blir sprikende mål, derfor må de deles inn: Det viktigste er
% plattformen, og i første omgang å få inn jevnt med penger sånn vi kan prøve
% oss på andre typer produkter. ALLE produktene vil styrke plattformen.
% Apple/Samsung gjør IKKE dette, og ignorerer dette helt (kan være av en
% grunn). De PRØVER feks med iPad på fly, men de får det ikke helt til.

% Forretningsmål

Målet er å tjene penger ved å selge produktet.
Vi bør ha ca 500-1000 kunder for å nå $10-20m$ i omsetning.
Må ha nok penger til å ha en del ansatte, minst 10 pluss support, selv om vi
satser på self-service.

Starte å selge i Norge, men vi tenker globalt fra første dag.
Vi skjønner kulturen i dette markedet. Kunne satset på Tyskland, med 90
millioner innbyggere (som er et av primærlandene til Apple).

Kan eventuelt selge til hvem som helst over nettet, men må deles i regioner,
også på infrastruktur.

Andre åpenbare mål som må tas med?

Et godt forretningsmål er jo å tjene penger på å tilby noe som folk trenger.
Mens Apple sitt fokus er på helse og kommunikasjon, er vårt fokus på trygghet,
for eksempel. For gamle, for barn, osv. (Se på den klokka som Stian fant).

% Grunner til at målene kan oppnås


\section{SWOT-analyse}

\subsection{Styrker}
Internt, eksternt
\begin{itemize}
  \item Billigere mobilenheter
  \item Lengre batteritid (kanskje!)
  \item Tettere kundeforhold
  \item Mer dynamisk (CPU, minne, disklagring)
  \item Bedre sikkerhet (fra stjeling av mobil)
\end{itemize}

\subsection{Svakheter}
Internt, eksternt
\begin{itemize}
  \item Energibruk radio er ukjent
  \item Dårlig mobildekning, må ha wifi da
  \item Latency
  \item Kostnad ifbm server
  \item Hvis serverne er nede er mobilen ubrukelig
  \item Hvorfor skal kundene velge oss? Vi må vite nøyaktig hva vi frir til.
\end{itemize}

\subsection{Muligheter}
\begin{itemize}
  \item Kulere, tynnere mobil
  \item Konkurrenter prøver å få til det vi begynner med (tynne mobiler)
  \item De sliter med cloudløsninger fordi de hele tiden må velge
  hybridløsninger (vi har alt på cloud fra begynnelsen av)
\end{itemize}

\subsection{Trusler}
\begin{itemize}
  \item Lett for de store å imitere, men de mangler da markedsprofil (feks
      Samsung må lage ny mobil men må markedsføre under Samsung-navnet, vi har
      eget navn som er ensbetydet med konseptet)
  \item De kan lett kopiere, de har allerede mange bra hybridløsninger. De kan
  flytte mer over til cloud, men har stor kundemasse allerede. Vi kan vokse med
  kundemassen og får nok en first-mover advantage.
\end{itemize}

% Dette her er ganske tynt foreløpig! Bare få inn alt her, så kan det
% finjusteres etterhvert.

\section{Kundesegment}


\section{Kundeverdi og verdiløfte}

Alt ligger i skyen: Telefonen er bare et vindu inn til dine apps.
Sikkerhet. Oppgradering skjer i stor grad på serversiden (mer lagringsplass? du
betaler for det du bruker, automatisk.. får regning, så enkelt er det. Ikke
vits å kjøpe mer plass, du bare betaler for det du bruker, dvs pris per gig per
måned. Mer CPU-kraft? Dette er også rullende, alt etter hvor mye du bruker. Så
hvis du kjører mandelbrot-program så koster det mer.

Komprimering mellom enhet og sky: All data kan i prinsippet komprimeres, og alt
streames. Dvs når du hører musikk streames den, stopper du den så stopper
streamen. Trenger ikke lagre ting lokalt. PÅ serversiden er en filkopiering
bare en refcount som økes. Dette gjelder også bilder og sånt, feks hvis man
deler et bilde til mange venner så er det bare snakk om å refcounte det ene
bildet.

Support, kan dele skjerm med supportfolk, så de kan hjelpe deg.

Husk at folk betaler ofte for ting som dropbox. Vår løsning har jo dette
integrert, for mobil-only altås, men sammenlignet med iDrive så er det del av
greia... må jo ha en del lagring gratis fra start.

\section{Distribusjonskanaler}

\section{Kunderelasjon}

\section{Inntektsstrøm}

\section{Nøkkelressurser og kritiske suksessfaktorer}


\section{Kjerneaktiviteter}

Salg, infrastruktur, utvikle apps, tilby utviklerplatform.

\section{Partnere}


\section{Kostnadsstruktur}

Skille mellom kostnad for å få opp alt og selge første telefon, deretter
operativ drift etter vi har kunder. Blir en overgang, hvor det brennes mye
penger i begynnelsen og så stabiliserer seg over tid. Over tid vil man også se
etter måter å redusere kostnader på, feks inngå volumavtaler, flytte apps
nærmere kunden (CDN) osv. Disse må være TALLFESTET.

Datasenter: Hvor mange brukere per maskin? Husk at vi i prinsippet tillater kun
1 app å kjøre til enhver tid. De som ikke er aktive saver state (i API, bør
være billig som i kreve lite lagringsplass).

Drift av sentre, båndbredde, voip, sms-gw, osv. Blir heavy på denne siden, a la
altibox. Må ha CDN osv.

R\&{}D blir heavy i begynnelsen, faktisk en syk kost for å få det til.

Telefon: Må beregne å ta mellom 2-3 tusen for den? Regna litt på det.

Må også se på inntekter... husk at datatrafikk blir en del, men vi MÅ over på
wifi. Ser for meg 2500 for telefon og hvis vi ikke har integrert dataabb så må
det koste 150 i mnd, dette på toppen av dataabb som blir fort 300. Så blir jo
egentlig ganske dyrt. Vi kan lure inn en avtale som egen operatør, men det blir
fremdeles dyrt. Target må være 300 kr per måned som tak, ellers vil jo folk
velge andre tjenester. Poenget er jo det å ha en dritbillig tlf.

ALTERNATIV: Lage en smartklokke som er tynnklient? Dritvanskelig å lage så små
devices, men den kan være ypperlig siden det er mindre skjerm = mindre data,
den trenger virkelig dette med å kunne offsette cpu-kraft til skyen. Dette er
også et marked som ikke fins enda, alle prøver seg, så det må være "lov" å
eksperimentere her.. fins jo en klokke (fred.olsen) som er always on (AT\&{}T),
får sikkert med telenor på dette.

Bruke mirasol display... e-ink, bruker lite strøm. Trenger ingen backlight. Kan
ses i sollys.

Pluss, folk er vel egentlig ikke interessert i å se video på små klokker, men
vil da heller bruek den til andre ting.. og vi kan tilby masse gudd shitt,
spesielt siden man ikke trenger mye lagringsplass... inkl sende og motta
beskjeder (via diktering, som skjer serverside), samtaler (enkelt å rute)..
tror ikke apple watch har noe mobilgreier i seg, så her har vi fordel.. MÅ være
integrert sim, så har et lite window of opportunity her.

Se på value proposation whitepaper for mirasol, står en del interessante tall
der som vi kan bruke.

Hvem vil ha denne da? Faktisk kan vi til og med selge til de som ikke vanligvis
bruker mobil. Gamle folk kan ha et kodeord de sier, så aktiveres en
hjelpefunksjon (feks operatør) som kan hjelpe dem. Kan brukes av klatrere,
redningsfolk.. vanlige folk som er på farten, osv. Eneste er integrasjon m
mobil som blir et problem, vi bygger jo egen silo her.

"Eventually we're going to make a mobile phone", sånn pitch-aktig.

USA er egentlig perfekt marked, for vi kan ha avtale på hele USA. I Europa er
det mer stress, men det har blitt gjort for enkle ting (feks TomTom).

Apple watch resolution: 340px height liten modell, 390px stor modell.
Så ca 270x340 vs 312x390
Det blir 270*340=92k vs 312*390=121k
Hvor mye kan PNG komprimere? GIF har typisk 4:1-10:1, PNG 10-30 prosent mindre,
En kombinasjon av JPEG+lossless funker og blir mye brukt... JPG har jo i medium
30-50:1, så vi kan regne med kanskje 1:10-1:20? da snakker vi om 12k-24k for et
bilde. Animasjoner: Kan ha swipe-animasjoner som default, sånn internt minne
bufrer 4 skjermer (for swiping i alle retninger), swiping er da smooth.
Fabrice bellard sin sak, BPG, gir vel en bra komprimering vs kvalitet?
Så hiver på komprimering, tipper vi kan nå 1/10, så 9-12k per bilde da.

Tror dette kan funke...

