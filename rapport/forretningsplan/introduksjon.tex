\section{Idé og visjon}

I prosjektskissen presenterte vi en visjon om å lage en \textem{tynnklient}
smarttelefon. Telefonen vil da bli en dum terminal som formidler interaksjon
til servere som kjører applikasjoner i skyen. Skjermbilder sendes ned til
telefonen.

Argumentet var at det ville lede til lavere produksjonskost, lengre
batteritid, høyere brukersikkerhet, samt at telefonhuset ville bli lettere og
mindre.  Imidlertid vil dette lede til høyere datatrafikk, og våre estimater
viser at det vil kreve et gjennomsnittlig månedlig forbruk på rundt 10 Gb.

Dette er fremdeles vår visjon, men tiden er ikke enda moden for slike
datamengder. På sikt vil sannsynligvis slike datamengder bli vanlige og dermed
billigere, men akkurat nå er dette for tidlig.

Istedenfor har vi gjort en såkalt \textem{pivot} --- en endring av
konseptet --- til å produsere \textem{tynnklient smartklokker}.  På grunn av
mindre skjermer vil man kunne tjene penger på en slik modell, og man får samme
fordeler. 

Smartklokker er et nytt marked hvor både store aktører som Apple og Samsung
satser tungt. Man har også mindre, innovative produsenter som Pebble, verdt omtrent \${}1.4-1.8 mrd i dag, m

Med introduksjon av smartklokker fra produsenter som Apple, Samsung, Pebble og
flere kan markedet vokse fra \${}1.4-1.8 milliarder idag til \${}10
milliarder i 2018 \cite{citi.grow}.

Dette er også et nytt marked med store aktører som Apple og Samsung
og mindre innovatører som Pebble og Tinitell. Man anslår at markedet vil vokse
fra NOK 10-14 milliarder i dag til 76 milliarder i 2018 \cite{citi.grow}.

