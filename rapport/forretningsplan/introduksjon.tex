\section{Idé og visjon}

I prosjektskissen presenterte vi en visjon om å lage en \textem{tynnklient}
smarttelefon. Telefonen vil da bli en dum terminal som formidler interaksjon
til servere som kjører applikasjoner i skyen. Skjermbilder sendes ned til
telefonen.

Argumentet var at det ville lede til lavere produksjonskost, lengre
batteritid, høyere brukersikkerhet, samt at telefonhuset ville bli lettere og
mindre.  Imidlertid vil dette lede til høyere datatrafikk, og våre estimater
viser at det vil kreve et gjennomsnittlig månedlig forbruk på rundt 10 Gb.

Dette er fremdeles vår visjon, men tiden er ikke enda moden for slike
datamengder. På sikt vil sannsynligvis slike datamengder bli vanlige og dermed
billigere, men akkurat nå er dette for tidlig.

Istedenfor har vi gjort en såkalt \textem{pivot} --- en endring av
konseptet --- til å produsere \textem{tynnklient smartklokker}.  På grunn av
mindre skjermer vil man kunne tjene penger på en slik modell, og man får samme
fordeler.

Dette er også et nytt marked med store aktører som Apple og Samsung
og mindre innovatører som Pebble og Tinitell. Man anslår at markedet vil vokse
fra NOK 10-14 milliarder i dag til 76 milliarder i 2018 \cite{citi.grow}.

Samtidig har ikke forbrukere etablert forventninger til denne
produktkategorien. Vi har dermed muligheten til å forme kategorien med vårt
unike konsept: En tynnklient smartklokke som er på nett hele tiden.

Dette produktet skiller seg ut fra mengden ved at

\begin{itemize}
  \item den er billigere
  \item den vil ha lengre batteritid
  \item forbrukeren blir en tjenesteabonnent, som leder til et tettere bånd
  mellom oss og kunden
  \item den kan kjøre vilkårlig avanserte applikasjoner, da dette kjøres i
  nettskyen
  \item den kan fjernstyres ved behov, for eksempel ved brukerstøtte
  \item den blir langt enklere å utvikle apps for, da man kun trenger å sende
  skjermbilder
  \item den kan spesialtilpasses nisjekategorier, som for eksempel
  trygghetsalarm for eldre eller sporingsenhet for barn
  \item den vil være sikrere, da data lagres i skyen
  \item oppdateringer vil skje automatisk i skyen
\end{itemize}

Den kortsiktige planen er å selge smartklokker. Den langsiktige planen er å
bygge op en teknologiplattform og patentportefølge som skal brukes til å selge
tynnklienttelefoner og -nettbrett.
