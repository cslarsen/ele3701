\section{Kostnadsstruktur}

Hele kostnadsstrukturen kan deles inn i to store faser: \textbf{Opprettelse av
bedrift}, altså utvikling av første fungerende produkt og etablering av
organisasjonell infrastruktur, og \textbf{Jevn drift}, skalering av salg og
minimering av kostnader.

For å se om bedriften er levedyktig må vi estimere kostnader så godt vi kan.
Under har vi gitt en forenklet \em{bill of materials}, eller \em{BOM}.
% TODO: Gi ref til hvordan vi har kommet frem til priser og volumpriser

\begin{table}[h]
  \centering
  \begin{tabular}{lrr}
    \textbf{Del} & \textbf{Pris (NOK)} & \textbf{Volumpris (NOK)} \\
    \hline
    CPU                  & 152.00        &  114.00 \\
    RAM                  &   7.60        &    3.80 \\
    ROM                  &   7.60        &    3.80 \\
    Batteri              &  15.20        &    7.60 \\
    Skjerm               &  76.00        &   60.80 \\
    I/O og lyd           &   7.60        &    6.08 \\
    Annen elektronikk    &  76.00        &   60.80 \\
    Ikke-elektronikk     &  76.00        &   60.80 \\
    Montering og testing &  15.20        &   15.20 \\
    \\
    \textbf{Sum}  & 433.20        &  332.88 \\
  \end{tabular}
  \caption{Produksjonskostnad per smartklokke}
  \label{table.pris.klokke}
\end{table}
% TODO: Gi referanser hvordan vi har kommmet frem til dette.

For å beregne datatrafikk har vi tatt utgangspunkt i en skjermstørrelse på
144x168 piksler, en komprimeringsfaktor\footnote{Komprimeringsalgoritmer har en
  typisk reduksjonsfaktor på $\sfrac{1}{10}$--$\sfrac{1}{20}$, men ved å
  utnytte typiske bruksmønstre kan vi optimalisere dette ved bruk av smart
mellomlagring.} på $\sfrac{1}{20}$ og sending av nytt skjermbilde hvert femte
sekund, hele døgnet.  Trafikkmengden per bruker per måned blir da
%
\begin{gather*}
144\cdot168\cdot\sfrac{1}{20}\cdot\sfrac{3600}{5}\cdot24\cdot31\cdot\sfrac{1}{10^9}
= 0.65~\sfrac{\textsf{Gb}}{\textsf{mnd}}
\end{gather*}

Vi anser dette som et øvre tak på trafikkmengden. Det betyr at kunder som har
eksisterende abonnement kan benytte tvilling-SIM. For de som ikke har det kan
vi selge eget SIM-kort og ta rundt kr 100 per måned.

%
For å beregne kostnader til serverpark har vi tatt utgangspunkt i en
konfigurasjon som skal takle minst 250 kunder. Kostnadene er per måned.
% TODO: HAr brukt AWS https://awstcocalculator.com/# for dette

\begin{table}[h]
  \centering
  \begin{tabular}{lr}
  \textbf{Server}      & \textbf{Pris (USD)} \\
  Tykk klient       & ? \\
  Voice og SMS      & ? \\
  Database          & ? \\
  Lagring (1 Tb)    & \\
  \textbf{SUM (NOK)} & 3200 \\
  \end{tabular}
  \caption{Serverpris per måned}
  \label{table.serverpris}
\end{table}

Tror vi kan klare oss med langt mindre maskiner. MÅ ha to maskiner per
brukermasse, men det tenker vi ikke på nå. Så 1 maskin per.

Enhet per måned blir da ca 30 kr og pris for server pr måned blir da 26.32.
For ett år: 675 kr. Tar vi 1000 kr per klokke og 100 kr i månede blir det da
2200 kr, fortjeneste på 1525. Datatrafikk har vi ikke medregnet!

Problemet er hvor mange kunder vi trenger. Fortjenest er ca 127 pr måned, så
1000 kunder gir en inntekt på 1.5 mill. Det er ikke i nærheten av hva vi
trenger. Trenger nok en 6-7 tusen kunder før vi når 10 mill i fortjeneste.
Per ansatt må vi nok regne med 1 mill, så dette er egentlig greit, gitt vi får
nok kunder. Husk vi mangler datatrafikk!

Vår strategi er å tilby tvillingsim, så for de som vil ha det kan de bruke
samme SIM i klokke og telefon. For de som har kun klokke blir det noenlunde
likt. Hvis det koster 100 kr pr måned bare å ha abb, og tak er 300 kr pr måned
for data, så har vi 200 kr å rutte med. Vi trenger mao kun å regne trafikk
mellom klokke og oss, altså impact på å ha oss. De fleste klarer seg fint på 6
Gb datatrafikk, kanskje 3 Gb også. Så tror regnestykket skal være greit.

\subsection{Notater}

Skille mellom kostnad for å få opp alt og selge første telefon, deretter
operativ drift etter vi har kunder. Blir en overgang, hvor det brennes mye
penger i begynnelsen og så stabiliserer seg over tid. Over tid vil man også se
etter måter å redusere kostnader på, feks inngå volumavtaler, flytte apps
nærmere kunden (CDN) osv. Disse må være TALLFESTET.

Datasenter: Hvor mange brukere per maskin? Husk at vi i prinsippet tillater kun
1 app å kjøre til enhver tid. De som ikke er aktive saver state (i API, bør
være billig som i kreve lite lagringsplass).

Drift av sentre, båndbredde, voip, sms-gw, osv. Blir heavy på denne siden, a la
altibox. Må ha CDN osv.

R\&{}D blir heavy i begynnelsen, faktisk en syk kost for å få det til.

Telefon: Må beregne å ta mellom 2-3 tusen for den? Regna litt på det.

Må også se på inntekter... husk at datatrafikk blir en del, men vi MÅ over på
wifi. Ser for meg 2500 for telefon og hvis vi ikke har integrert dataabb så må
det koste 150 i mnd, dette på toppen av dataabb som blir fort 300. Så blir jo
egentlig ganske dyrt. Vi kan lure inn en avtale som egen operatør, men det blir
fremdeles dyrt. Target må være 300 kr per måned som tak, ellers vil jo folk
velge andre tjenester. Poenget er jo det å ha en dritbillig tlf.

ALTERNATIV: Lage en smartklokke som er tynnklient? Dritvanskelig å lage så små
devices, men den kan være ypperlig siden det er mindre skjerm = mindre data,
den trenger virkelig dette med å kunne offsette cpu-kraft til skyen. Dette er
også et marked som ikke fins enda, alle prøver seg, så det må være "lov" å
eksperimentere her.. fins jo en klokke (fred.olsen) som er always on (AT\&{}T),
får sikkert med telenor på dette.

Bruke mirasol display... e-ink, bruker lite strøm. Trenger ingen backlight. Kan
ses i sollys.

Pluss, folk er vel egentlig ikke interessert i å se video på små klokker, men
vil da heller bruek den til andre ting.. og vi kan tilby masse gudd shitt,
spesielt siden man ikke trenger mye lagringsplass... inkl sende og motta
beskjeder (via diktering, som skjer serverside), samtaler (enkelt å rute)..
tror ikke apple watch har noe mobilgreier i seg, så her har vi fordel.. MÅ være
integrert sim, så har et lite window of opportunity her.

Se på value proposation whitepaper for mirasol, står en del interessante tall
der som vi kan bruke.

Hvem vil ha denne da? Faktisk kan vi til og med selge til de som ikke vanligvis
bruker mobil. Gamle folk kan ha et kodeord de sier, så aktiveres en
hjelpefunksjon (feks operatør) som kan hjelpe dem. Kan brukes av klatrere,
redningsfolk.. vanlige folk som er på farten, osv. Eneste er integrasjon m
mobil som blir et problem, vi bygger jo egen silo her.

"Eventually we're going to make a mobile phone", sånn pitch-aktig.

USA er egentlig perfekt marked, for vi kan ha avtale på hele USA. I Europa er
det mer stress, men det har blitt gjort for enkle ting (feks TomTom).

Apple watch resolution: 340px height liten modell, 390px stor modell.
Så ca 270x340 vs 312x390
Det blir 270*340=92k vs 312*390=121k
Hvor mye kan PNG komprimere? GIF har typisk 4:1-10:1, PNG 10-30 prosent mindre,
En kombinasjon av JPEG+lossless funker og blir mye brukt... JPG har jo i medium
30-50:1, så vi kan regne med kanskje 1:10-1:20? da snakker vi om 12k-24k for et
bilde. Animasjoner: Kan ha swipe-animasjoner som default, sånn internt minne
bufrer 4 skjermer (for swiping i alle retninger), swiping er da smooth.
Fabrice bellard sin sak, BPG, gir vel en bra komprimering vs kvalitet?
Så hiver på komprimering, tipper vi kan nå 1/10, så 9-12k per bilde da.

Tror dette kan funke...

For å være kul kan man også sende vektorisert gfx, men tror egentlig det
koster mer cpu kraft enn det smaker.

ANimasjon: gif=1892kb, bpg=48kb, svaings
