\section{Kostnadsstruktur}

Skille mellom kostnad for å få opp alt og selge første telefon, deretter
operativ drift etter vi har kunder. Blir en overgang, hvor det brennes mye
penger i begynnelsen og så stabiliserer seg over tid. Over tid vil man også se
etter måter å redusere kostnader på, feks inngå volumavtaler, flytte apps
nærmere kunden (CDN) osv. Disse må være TALLFESTET.

Datasenter: Hvor mange brukere per maskin? Husk at vi i prinsippet tillater kun
1 app å kjøre til enhver tid. De som ikke er aktive saver state (i API, bør
være billig som i kreve lite lagringsplass).

Drift av sentre, båndbredde, voip, sms-gw, osv. Blir heavy på denne siden, a la
altibox. Må ha CDN osv.

R&D blir heavy i begynnelsen, faktisk en syk kost for å få det til.

Telefon: Må beregne å ta mellom 2-3 tusen for den? Regna litt på det.

Må også se på inntekter... husk at datatrafikk blir en del, men vi MÅ over på
wifi. Ser for meg 2500 for telefon og hvis vi ikke har integrert dataabb så må
det koste 150 i mnd, dette på toppen av dataabb som blir fort 300. Så blir jo
egentlig ganske dyrt. Vi kan lure inn en avtale som egen operatør, men det blir
fremdeles dyrt. Target må være 300 kr per måned som tak, ellers vil jo folk
velge andre tjenester. Poenget er jo det å ha en dritbillig tlf.

ALTERNATIV: Lage en smartklokke som er tynnklient? Dritvanskelig å lage så små
devices, men den kan være ypperlig siden det er mindre skjerm = mindre data,
den trenger virkelig dette med å kunne offsette cpu-kraft til skyen. Dette er
også et marked som ikke fins enda, alle prøver seg, så det må være "lov" å
eksperimentere her.. fins jo en klokke (fred.olsen) som er always on (AT&T),
får sikkert med telenor på dette.

Bruke mirasol display... e-ink, bruker lite strøm. Trenger ingen backlight. Kan
ses i sollys.

Pluss, folk er vel egentlig ikke interessert i å se video på små klokker, men
vil da heller bruek den til andre ting.. og vi kan tilby masse gudd shitt,
spesielt siden man ikke trenger mye lagringsplass... inkl sende og motta
beskjeder (via diktering, som skjer serverside), samtaler (enkelt å rute)..
tror ikke apple watch har noe mobilgreier i seg, så her har vi fordel.. MÅ være
integrert sim, så har et lite window of opportunity her.

Se på value proposation whitepaper for mirasol, står en del interessante tall
der som vi kan bruke.

Hvem vil ha denne da? Faktisk kan vi til og med selge til de som ikke vanligvis
bruker mobil. Gamle folk kan ha et kodeord de sier, så aktiveres en
hjelpefunksjon (feks operatør) som kan hjelpe dem. Kan brukes av klatrere,
redningsfolk.. vanlige folk som er på farten, osv. Eneste er integrasjon m
mobil som blir et problem, vi bygger jo egen silo her.

"Eventually we're going to make a mobile phone", sånn pitch-aktig.

USA er egentlig perfekt marked, for vi kan ha avtale på hele USA. I Europa er
det mer stress, men det har blitt gjort for enkle ting (feks TomTom).

Apple watch resolution: 340px height liten modell, 390px stor modell.
Så ca 270x340 vs 312x390
Det blir 270*340=92k vs 312*390=121k
Hvor mye kan PNG komprimere? GIF har typisk 4:1-10:1, PNG 10-30 prosent mindre,
En kombinasjon av JPEG+lossless funker og blir mye brukt... JPG har jo i medium
30-50:1, så vi kan regne med kanskje 1:10-1:20? da snakker vi om 12k-24k for et
bilde. Animasjoner: Kan ha swipe-animasjoner som default, sånn internt minne
bufrer 4 skjermer (for swiping i alle retninger), swiping er da smooth.
Fabrice bellard sin sak, BPG, gir vel en bra komprimering vs kvalitet?
Så hiver på komprimering, tipper vi kan nå 1/10, så 9-12k per bilde da.

Tror dette kan funke...
