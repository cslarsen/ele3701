\chapter{Ny tekst}

\section{Forretningsidé}

Når iPhone ble introdusert i 2007 så vi raskt behovet for mobile enheter som
kunne kjøre vilkårlig programvare. Det viktigste var kanskje at de alltid var
koblet til Internett.

Samtidig har det vært en tydelig trend mot \textit{virtualisering} av
maskinvare --- altså fysisk maskinvare som erstattes av programvare.  I dag vil
de nettsteder starte og stoppe virtuelle tjenere i takt med arbeidsbehovet.

Imidlertid har vi enda ikke sett virtualisering av mobile enheter.

Produsenter som Apple og Samsung konkurrerer hardt med å levere telefoner og
smartklokker som er kraftigere, tynnere og har lengre batteritid.
%
Konsumentene har på sin side behov for økt sikkerhet og bedre støttetjenester.
%
Alle disse behovene kan løses med virtualisering av den mobile enheten, hvor
kompleksiteten flyttes over til tjenere i skyen.

Vår forretningsidé er å tilby en totalløsning for mobile tynnklienter.

For å redusere

For å redusere omfanget skal vi i første omgang utvikle en virtuell
trygghetsalarm sammen med partnere i dette nisjemarkedet. Trygghetsalarmer i
dag krever en hussentral med SIM-kort, og har svært begrenset funksjonalitet.
Vår løsning vil være en enhet som bæres på armen eller rundt halsen, og vil
fungere over alt. Samtidig kan vi tilby 

I første omgang skal vi utvikle ferdigkomponenter som selges til
nisjeprodusenter. Disse skal da kunne styre disse med programvare som kjører på
vår tjenerplattform. Denne plattformen tilbyr enkle grensesnitt for å 
telefonsamtaler, sende meldinger og lignende.
Vi skal utvikle ferdigkomponenter som skal selges til nis
Vi skal utvikle ferdigkomponenter som styres fra 


IT-industrien har historisk sett
- beskriv hvilket behov i markedet som skal dekkes
  - BEHOV for tynnklientløsninger som kan fullstendig tilpasses som
  nisjekategorier
  - BEHOV for trygghet, sikkerhet
    - Sveits er nok interessert i smartklokkekomponenter
    - Kan også selges som komponenter til vending machines osv (kaffeautomater,
        osv)
    - er en generell trend at alt blir virtualisert likevel.. har alltid gått
    fra spesifikk harware, generell hardware, networking, og til slutt
    virtualisering.

  - tynnklient = paradigmeskifte
    - "Se for deg et tastatur, skjerm, høyttalere og mikrofon med
    usynlige ledninger som går til serverne våre. Det er dette
    tynnklientkonseptet handler om."
  - billigere, lengre batteritid
  - sikkerhet
    - ingen oppgradering
    - ingen installasjon
    - all data sikret i skyen
    - backup ikke nødvendig
  - tynnklient leder til nye bruksområder
    - spesialisering av hele produktet, med samme hardware.
      - endrer kun på serversoftware
      - kan da selge egne produkter som trygghetsalarm, kommunikasjonsløsninger
      for barn (tracking, voice, ingen spill på enheten)
  - andre behov
    - mye sterkere måte å beskytte bruk for barn
      - sites man ikke skal besøke, apps man ikke skal bruke
      - tidsstyring
      - det er _umulig_ å komme rundt dette
    - samme for eldre
    - kan sannsynligvis også brukes i andre nisjekategorier, feks
      - løsning for industriarbeidere som trenger frie hender, har bluetooth
      headset og klokke på armen, kan gi kommunikasjon..
      - feks talestyring for plukking på BAMA (de har kjøpt et dyrt system, men
          dette er _spesialisert_ for dette ene behovet; vi lager en PLATTFORM
          der slike ting kan gjøres)
    - kan også selges som vanlig klokke, men tror vi skal styre unna dette
    foreløpig, mens vi bygger opp
    - på sikt: smartklokke, mobiltelefon, tablet
    - kan også bli komponentleverandør, som kanskje er mer realistisk

- beskriv kundesegmentet som har dette behovet
  - eldre
  - barn
  - industriarbeidere

- beskriv den unike spisskompetansen bedriften har for å kunne dekke behovet.


