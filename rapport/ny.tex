\chapter{Ny tekst}

\section{Forretningsidé}

Når iPhone ble introdusert i 2007 så vi raskt behovet for mobile enheter som
kunne kjøre vilkårlig programvare. Det viktigste var kanskje at de alltid var
koblet til Internett.

Samtidig har det vært en tydelig trend mot \textit{virtualisering} av
maskinvare --- altså fysisk maskinvare som erstattes av programvare.  I dag vil
de nettsteder starte og stoppe virtuelle tjenere i takt med arbeidsbehovet.

Imidlertid har vi enda ikke sett virtualisering av mobile enheter.

Produsenter som Apple og Samsung konkurrerer hardt med å levere telefoner og
smartklokker som er kraftigere, tynnere og har lengre batteritid.
%
Konsumentene har på sin side behov for økt sikkerhet og bedre støttetjenester.
%
Alle disse behovene kan løses med virtualisering av den mobile enheten, hvor
kompleksiteten flyttes over til tjenere i skyen.

Vår forretningsidé er å tilby en totalløsning for mobile tynnklienter.

For å redusere omfanget og komme raskt på markedet skal vi i første omgang
utvikle en trygghetsalarm basert på denne totalløsningen.
%
Dette gjøres sammen med en etablert partner i dette nisjemarkedet.
%
Alarmen bæres på armen eller rundt halsen og vil alltid være koblet opp mot
nettet.
%
Mens eksisterende produkter krever en hussentral vil vår løsning kunne brukes
både hjemme og ute.
%
Alarmen blir spesialtilpasset formålet ved at den kun kan styres med et
spesifikt program på tjenersiden.
%
På denne måten kan vi tilby billige, energieffektive og brukervennlige
produkter i en rekke nisjekategorier.

Mens plattformen blir mer moden vil vi på lang sikt tilby tilsvarende
tynnklientløsninger for mobiltelefoner, smartklokker og nettbrett.
