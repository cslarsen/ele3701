\chapter{Ny tekst}

\section{Forretningsidé}

Etter iPhone ble introdusert i 2007 så vi raskt behovet for mobile enheter som
kunne kjøre vilkårlig programvare. Det viktigste var kanskje at de alltid var
koblet til Internett.

Samtidig har det vært en tydelig trend mot \textit{virtualisering} av
maskinvare --- fysisk maskinvare som erstattes av programvare
\cite{2006.virtualization.trends}.  For eksempel vil nettsteder i dag starte
opp og stenge ned virtuelle tjenere i takt med trafikkmengden, og stasjonære
skrivebordsmaskiner kan \textit{leies} over nettet gjennom
tynnklientgrensesnitt.

Imidlertid har det vært lite fremgang på virtualisering av mobile enheter
\cite{embedded.virtualization}.

Produsenter som Apple og Samsung konkurrerer hardt med å levere enheter som er
kraftigere, tynnere og har lengre batteritid.
%
Konsumentene har på sin side behov for økt sikkerhet og bedre
tilpasningsmuligheter.
%
Videre finnes det nisjeprodusenter som kun ønsker å benytte deler av
funksjonaliteten til smarttelefoner.
%
Man ønsker gjerne å styre enheten fullt ut, slik at den fremstår som et separat
produkt.

Disse problemene kan løses med virtualisering av den mobile enheten, hvor
kompleksiteten flyttes over til tjenere i skyen \cite{mobil.virt.fordel}, og
enheten står igjen som en dum terminal --- en tynn klient --- som
videreformidler bilde, lyd og interaksjon.

\textbf{Vår visjon er å levere en totalløsning for mobile tynnklienter,
bestående av maskinvare og en tjenerplattform.}

For å realisere denne visjonen skal vi redusere omfanget og fokusere på et
spesifikt produkt: En trygghetsalarm for eldre.

Eksisterende trygghetsalarmer krever en dyr hussentral for å være på nett.
Dermed kan alarmen kun brukes i hjemmet, med mindre man husker å ta med en
tilkoblet mobiltelefon når man skal på tur. De kan heller ikke tilpasses
brukeren. Begynnende demente har samme funksjonalitet som friske personer som
kun er redde for fall. Man kan heller ikke endre oppførselen etter at alarmene
er sendt i posten.

Vår løsning vil derimot være helt selvstendig og kunne brukes overalt. Det blir
dermed et billigere alternativ som ikke krever en sentral. Vi kan også tilpasse
den individuelle brukeren, eller når han får behov for ny funksjonalitet. Man
vil kunne innhente anonyme bruksmønstre for å se hva som fungerer godt, og
oppdatere programvaren tilsvarende. Pårørende vil også ha muligheten til å
følge brukerens lokasjon.

\textbf{Forretningsidéen er dermed å utvikle vår tynnklient mobilplattform
gjennom å lage en trygghetsalarm som første produkt.}

På lang sikt ønsker vi å utvikle tilsvarende enheter for mobiltelefoner,
smartklokker og nettbrett.

\section{SWOT-analyse}

Har forsåvidt forklart hva fordelene våre er, må inn i denne matrisen.

\begin{table}[h]
  \begin{tabular}{lll}
  & \textbf{Styrker}
  & \textbf{Svakheter} \\
  Internt
  & ...
  & ...
  \\
  & \textbf{Muligheter}
  & \textbf{Trusler} \\
  Eksternt
  & ...
  & ...
  \\
  \end{tabular}
\end{table}

\section{Forretningsmodell}

\subsection{Kundesegment}

Kundene våre er produsenter av sluttbrukerenheter med behov for et
\textit{subsett} av smarttelefonens funksjonalitet. De har typisk behov for å
koble seg opp til internett, telekommunikasjon, innhenting av data fra sensorer
og bruksmønstre, fjernstyring og så videre.

Samtidig sikter vi oss inn på norske produsenter som i stor grad kjøper
ferdigkomponenter som settes sammen til et produkt.  Når vi har operativ drift
og bedriften er moden skal vi gå ut på det internasjonale markedet.

Denne kundeprofilen vil legge vekt på hvor enkelt det er å utvikle for en slik
komponent. Med oss blir mye av den maskinvaremessige delen erstattet med
utvikling av programvare på vår skyplattform.

Kjøpskriterier: Vi må ta en månedlig avgift for drift en nettskyen. Dette må
passe med kundens produkt. For trygghetsalarmer tar man allerede en
månedsavgift for produktet, så det skal passe bra.

% spesialtilpasning
% Dignio
% pris? funksjonalitet i hardware
% hvilke behov har deres kunder da? altså de som har trygghetsalarm?
% de har behov for en del sensorer også, og der kommer jo husalarmen inn. men
% det skal ikke vi ha... kan muligens ha bt tilgang til sensorer, feks ser lett
% om man åpner kjøleskapet med klokken.
% TINITELL

% Se for meg webside "We offer a hardware component to build smart appliances
% and a cloud OS to run specialize thin client stuff. Our success stor is the
% trygghetsalarm for blabla..."

% Behov: Internett, enkel utvikling, spesialisering, fjernstyring,
% konfigurasjon, datainnhenting, tilpasning, slippe å utvikle alt selv
% Hvem er kundene: Produsenter av sånne type produkt
% Hva vektlegger kunden ved valg av leverandør/tjeneste: Enkelhet, pris,
% seriøshet, hvordan koble til annen harwdare/sensorer... beliggenhet, vi er jo
% norske..

\subsection{Kundeverdi og verdiløfte}

% Hva er bedriftens unike styrke i konkurransen om kundene? Fins det lignende
% produkt egentlig? vår unike styrke er at vi tilbyr en totalløsning for tynne
% klienter, altså både hardware OG ferdig software. Kunden trenger bare
% programmere i prinsippet for å lage produkt. Vårt produkt kan også utvides,
% er alltid oppdatert, alltid online. Sikker lagring av informasjon.

% VÅRE KONKURRENTER (passser ikke inn her), de fleste driver thin client på
% desktoppen.. Vi vil tilby det på embedded systems, sammen med tilhørende
% cloud OS.

% Hva tilbyr bedriften sine kunder? Hardware-komponent, thin-client cloud OS

% HVilke verdier skaper bedriften for sine kunder (verdiløftet): Se over,
% fjernstyring, innhenting av brukermønstre, ferdig os å utvikle for. Hva har
% vi av verdi? Jo, at kjedelige produkter kan "internett-enables". Det er jo
% hele greia: At produkter som tidligere har hatt vanskelig for å komme på nett
% og med begrenset funksjonalitet nå kan gjøres ekstremt lett.

% Hmm, dette kalles visst "zero client", eller "ultathin client"., eller nei,
% de snakker om vdi, men det alle snakker om er mobiltlf som desktop thin
% client. de bommer litt her.

\subsection{Distribusjonskanaler}

For denne type produkt blir det mye snakk på telefon, sending av deler på epost
for prototyping, det blir mange salgsmøter og demonstrasjoner. Det blir tett
partnersamarbeid. Det vi ønsker å gi til kunden er muligheten til å lage et
bedre produkt samtidig som de sparer tid og penger.

For å markedsføre oss må vi ha en bra profil på nett som er søkbar, men vi må
også være tilstede på bransjekonferanser, vi må bli profilert i bransjeblader.
Helst må vi få en tilstedeværelse på nett. Når tiden er inne og vi skal gå
globalt skal vi være der at det er kjempeenkelt for eksterne å bruke vårt
produkt. Da kan det være tid for crowdsourcing. Ønsker å bli kjent for
utviklere som bruker Android for tilpasning, Arduino-folk osv. Litt hipt, ungt
image for å fakke folk på den nye plattformen. TechCrunch.

\subsection{Kunderelasjon}

I nesten alle tilfeller blir det snakk om partnerutvikling. Tror aldri vi
slipper unna dette, så det kan bli vanskelig å skalere så lenge vi er på
komponentnivå.

Dette er en bransje hvor det er vanskelig å skalere, for det krever mange folk.
Vi må dermed satse på at kunden i stor grad skal klare seg selv, men det skjer
kun på sikt.

Konkurrenter blir andre som er i relatert marked. Største konkurrent blir
giganter som Qualcomm. Fordelen vår er at vi har en skyplattform som Qualcomm
sannsynligvis aldri er interessert i å ha. Faktisk kan de bli en partner, der
vi bruker mye Qualcomm-komponenter i vår pakke, men med vår mer ferdige profil
og vår plattform som styrer disse.

Vår skyplattformen vil kundene våre ha programvare som kjører i skyen. Derfor
må vi her ha en enkel måte for self-service. A la google analytics, Amazon AWS
dashboard. Vi må ta månedlig avgift per device, eller per CPU og nettbruk.
Huff, er ikke helt dette jeg så for meg. Men er helt klart et marked for at
folk lager smart-appliances selv, og da gjennom oss, helt klart. Da blir jo
konkurrent fort Apple og Samsung, for de kan kobles opp til mange typer, de har
jo homekit og sånt og ser for seg å styre dette fra mobilen.
Men sånn sett så vil vi ligge godt under radaren lenge.
