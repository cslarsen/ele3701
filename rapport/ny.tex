% Alta forventer vekst på omtrent 50% personer over 67 i 2010-2020 og
% tilsvarende for de over 80 i 2020-2030.
% http://www.mynewsdesk.com/no/telenor/pressreleases/knytter-trygghetsalarm-til-pasientjournal-907498
% Fakta: Eldrebølgen, nordmenn over 67 øker fra 650k i dag til 1 mill før 2030
% Norge bruker 250 mrd på helse og sosialsektoren, eller 12% av BNP
% Utgiftene til sykehjemsplass er 850k i året, hjemmeboende er 200k
% SSB sykehjem øker fra 43k til 65k i 2030
% Sparing med velferdsteknologi: hvis 15-25% kan bo lengre hjemme sparer man
% 12-20mrd i 2030. Eller 40k årsverk som kan frigjøres.

% Sammendrag: Se
% http://nutsandbolts.mit.edu/Virtual_Ink/VirtualInk_Complete.pdf
% Sykt bra sammendrag... spesielt hvor mye penger vi trenger for å lage
% prototype.

% Se her!
% http://www.lister.no/phocadownload/helsenettverk_lister/2014_10_PA_rapport_om_alarmmottak.pdf
% Skrevet på oppgrad av helsedirektoratet... ser ut som SOS International er
% største leverandør. Det er et dansk firma tror jeg.

% GPS-skriv/studie fra SINTEF, de bruker safecall, for demens
% Kan ringe opp og lytte uten at noen vet det!
% http://www.sintef.no/globalassets/upload/konsern/trygge-spor-rapport_enkle-sider_lav-opplosning-2.pdf

% GPS: Lov å bruke GPS etter 2013 for denne typen tjenester, se
% https://www.stortinget.no/no/Saker-og-publikasjoner/Vedtak/Beslutninger/Lovvedtak/2012-2013/vedtak-201213-062/

% VIKTIG: Ta med i SKISSEN, at vi trenger å finne et passsende første produkt
% som bruker vår teknolog, slik det henger sammen med det som kommer etterpå.

% Første setning i sammendrag bør være ekstremt fengende: "Dagens
% mobilløsninger er avleggs, for de utnytter ikke no-install/nyeste versjon NÅ
% som man er vant med på internett <- gjelder altså fysiske devicer"

% Om produkt: Viktig å si "konkurrente kan IKKE endre funksjonalitet uten å
% danne en gruppe som skal sette seg ned og utvikle en prototype, fikse alle
% feil til de får det bra nok, søke godkjenninger, produsere selve driten og
% SÅ, etter ett år, kan de begynne å selge på markdet. Vi kan begynne å lage ny
% funksjonalitet på fredag og gjøre det tilgjengelig over nettet for ALLE
% brukerne våre på mandag."

% LES DENNE:
% http://www.uio.no/studier/emner/matnat/sfe/ENT4000/v12/undervisningsmateriale/04%20Forretningsplan%20Hansen.pdf
% Har glemt helt ut nettverk, det er viktig! Hva med konferanser, foredrag?
% Kapital, trnger vi ofte seff. Kjempegode tips her, spesielt grafisk eksempel.
% Redusere teknisk språk, spesielt i introen. Viktig med skalerbarhet! (det har
% vi, men må gjøres tydelig, spesielt VIS at vi har transformert problemet til
% å utvikle ny software, vi utviklet hardware bare én gang).
% Lyst å nevne planned obsolesence noe sted, hvor det passer.

% DIVERSE NOTATER
%
% Nitec, fins i stavanger, nitec.no. Kjempebra partner for utvikling.
% De selger faktisk trygghetsalarm med gsm i, men ser ut til å være stasjonær.
% Sier at analoge linjer skal legges ned, derfor trengs nye løsninger! <--- MÅ
% TAS MED! Opps deres løsning kan faktisk progges over nett, men har ikke
% display. Ser ut som de kjøper av caretech, gjør dem mer aktuell som partner.
% Skal vi selge til dem eller direkte? Vi ønsker å selge så direkte som mulig.
% Skal ha vertikalisering. Caretech er et svensk firma tydeligvis.
% De har gsm sak, men ser ut som en stasjon likevel, og små sensorer på
% armbånd, men de har falldetektor. Da må vi bare ha en liten støtsensor, men det
% koster jo litt med utviklingen. Ser også de tilbyr en rekke sensorer som man
% kobler opp mot sentralen, derav denne typen løsning. Dette kan vi tilby på
% sikt, men da at hver sensor er selvstendig, for eksempel. eller samma det.
% Hver bruker har forskjellige behov. Vet at mormor til Siri kun hadde
% alarm på hånden. De har knapp, lyser når den tar kontakt og blinker når den
% har kontakt. Så det er faktisk ikke rett fram det heller.De nevner overfall,
% innbrudd, personalknapp. Har batteriovervåkning, dette må vi også ha. Har
% tydeligvis batteri og kan ikke lades. Det er eneste ulempe med oss, må lade
% den hver natt. Kan være noen glemmer å ta den på. Løses ved at den sjekker om
% det er morgen, feks se når brukeren vanligvis står opp, og gir beskjed med
% hyling om den ikke er tatt på. :) De kjører IP67, så vi har jo en utfordring
% der, men det er bare vannavstøtende og ikke vanntett.
%
% Herlighet, se på lfh.no. Det er jo et sykt bra poeng at eldregruppen bare
% øker og øker, og vi forventer en eksplosjon på det området nå. Alle prognoser
% viser at det forventes sykt mye mer nødvendighet for hjelp. lfh.no sier i
% http://lfh.no/wp-content/uploads/2013/09/LFH-Tallfakta-kommune-WEB.pdf at man
% forventer at 380k personer vil ha behov for hjemmetjenester i 2040.
% Nivå for driftskostnader var 85 milliarder kr i 2012... Forventet i bli 147
% mrd i 2040. "Konsekvensene av eldrebølgen forventes å eskalere i 2020..."
% sier den saken. Derfor anbefaler LFH at Staten investerer i innovative
% løsninger allerede i dag.
% Hjemmetjenester er en del av primærhelsetjenesten i Norge (sykehus, osv).
% Vi opererer i markedet for hjemmetjenester.
%
% Konkurrenter: Moreto, gjort akkurat det vi vil, men på en annen måte
% http://www.tu.no/it/2013/01/27/nordmenn-gjor-trygghetsalarmen-tryggere
% Vi må omposisjonere oss litt pga dette. Foreslår biligere device, men dyrere
% månedsabb (de har andre priser for bedrift). Spørs om vi ønsker å konkurrere
% på pris, for det vi vil er å få enheten ut, for den kan kanskje brukes til
% BPA også.
%
% Se også skjema fra SINTEF, som bør bli partner:
% https://www.safemate.no/wp-content/uploads/2015/04/Skjema-for-vurdering-av-lokaliseringsteknologi-Safemate.pdf
%
% Må også satse på IP67 og A-GPS om en går på features.
%
% Se http://www.energibransjen.no/default.asp?menu=2&id=3908
% Moreto EDB gikk konkt, så kjøpte Lyse Smart konkursboet.
% Utfasing skjer 2017 og innen da må alle kommuner ha kjøpt nye alarmer som er
% på digitalt nett. Vår bør kunne kjøre på wifi også, forøvrig, og ha
% bluetooth, bare for å være bedre.
%
% Se her:
% http://www.ivarjohansen.no/temaer/sosialpolitikk/5198-trygghetsalarm-ogsa-for-fattigfolk-nei-bare-dersom-du-er-kredittverdig.html
% Kunne vært kult å subsidiere til slike personer, via firmaet selv?

% 110-telemark, se
% http://www.konkurransetilsynet.no/iKnowBase/Content/422277/A2006-44%20-%20Hjelp%2024%20Trygghetsalarm%20AS.pdf
% Klage fra hjelp24. Ble avslått.

% Mambo2 fra securinett tilbyr noe mer a la det vi vil lage
% http://www.securityworldhotel.com/no/Nyheter/Produktnyheter/securinet-presenterer-ny-trygghetsalarm#.VU0Bpc5K7i4
% Ser drit ut, men er der jaffal. Er DYR.

% Aleris vant kontrakt med telenor,
% http://www.nhoservice.no/article.php?articleID=5475&categoryID=329
% Viktigste er at det står at monopolet skal oppheves der i Oslo.

% Er mye konkurranse her, ass!

\chapter{Forretningsplan}

Mens vi i prosjektskissen tok utgangspunkt i \textit{The Business Model Canvas}
\cite{osterwalder} vil vi i denne forretningsplanen bruke malen fra Innovasjon
Norge \cite{innovasjon.norge}.

\textit{Merk at etter levering av prosjektskissen har vi valgt å fokusere på
trygghetsalarm som et konkret produkt (se begrunnelsen i kapittel
\vref{prosessen}).}

\section{Forretningsidé}

Etter iPhone ble introdusert i 2007 så vi raskt behovet for mobile enheter som
kunne kjøre vilkårlig programvare. Det viktigste var kanskje at de alltid var
koblet til Internett.

Samtidig har det vært en tydelig trend mot \textit{virtualisering} av
maskinvare --- fysisk maskinvare som erstattes av programvare
\cite{2006.virtualization.trends}.  For eksempel vil nettsteder i dag starte
opp og stenge ned virtuelle tjenere i takt med trafikkmengden, og stasjonære
skrivebordsmaskiner kan \textit{leies} over nettet gjennom
tynnklientgrensesnitt.

Imidlertid har det vært lite fremgang på virtualisering av mobile enheter
\cite{embedded.virtualization}.

Produsenter som Apple og Samsung konkurrerer hardt med å levere enheter som er
kraftigere, tynnere og har lengre batteritid.
%
Konsumentene har på sin side behov for økt sikkerhet og bedre
tilpasningsmuligheter.
%
Videre finnes det nisjeprodusenter som kun ønsker å benytte deler av
funksjonaliteten til smarttelefoner.
%
Man ønsker gjerne å styre enheten fullt ut, slik at den fremstår som et separat
produkt.

Disse problemene kan løses med virtualisering av den mobile enheten, hvor
kompleksiteten flyttes over til tjenere i skyen \cite{mobil.virt.fordel}, og
enheten står igjen som en dum terminal --- en tynn klient --- som
videreformidler bilde, lyd og interaksjon.

\textbf{Vår visjon er å levere en totalløsning for mobile tynnklienter,
bestående av maskinvare og en tjenerplattform.}

For å realisere denne visjonen skal vi redusere omfanget og fokusere på et
spesifikt produkt: En trygghetsalarm for eldre.

Eksisterende trygghetsalarmer krever en dyr hussentral for å være på nett.
Dermed kan alarmen kun brukes i hjemmet, med mindre man husker å ta med en
tilkoblet mobiltelefon når man skal på tur. De kan heller ikke tilpasses
brukeren. Begynnende demente har samme funksjonalitet som friske personer som
kun er redde for fall. Man kan heller ikke endre oppførselen etter at alarmene
er sendt i posten.

Vår løsning vil derimot være helt selvstendig og kunne brukes overalt. Det blir
dermed et billigere alternativ som ikke krever en sentral. Vi kan også tilpasse
den individuelle brukeren, eller når han får behov for ny funksjonalitet. Man
vil kunne innhente anonyme bruksmønstre for å se hva som fungerer godt, og
oppdatere programvaren tilsvarende. Pårørende vil også ha muligheten til å
følge brukerens lokasjon.

Med eldrebølgen forventer vi å måtte bruke mye mer tid og penger på
helsetjenester, spesielt hjemmetjenester \cite{lfh.innspill}. Teknologien i
dette markedet er fremdeles underlegen sammenlignet med en typisk smarttelefon
\cite{alarmparadokset}. Dette fører til at driftskostnadene er høyere enn det
kunne vært mer ny teknolog. Man forventer at driftsomkostninger for
primærhelsetjenesten skal øke fra 85 milliarder kroner i 2012 til 142
milliarder i 2040 \cite{lfh.innspill}.

\textbf{Forretningsidéen er dermed å utvikle vår tynnklient mobilplattform
gjennom å lage en avansert og driftsbillig trygghetsalarm som første produkt.}

Forretningsplanen vil være utformet med trygghetsalarmen som utgangspunkt, med
tynnklientløsningen som bakteppe. Vi vet hvor vi vil, men må ha en realistisk
strategi for å oppnå visjonen vår.

Etter hvert kan vi ta for oss andre typer nisjeprodukter som passer inn på vår
plattform.  På lang sikt ønsker vi å utvide tynnklientkonseptet til å omfatte
formfaktorer som mobiltelefoner, smartklokker og nettbrett. 

\section{SWOT-analyse}

% Kan godt skrive mer utfyllende og ikke stikkordsform. MÅ ikke være kvadratisk
% satt opp.

\begin{table}
  \centering
  \begin{tabular}{lll}
    \textit{Internt}
             & \textbf{Styrker}       & \textbf{Svakheter}            \\
             & Brukertilpasning       & Kompleks totalløsning         \\
             & Oppdatering            & Lite batteri                  \\
             & Enkelt oppsett         & Bemanning brukerstøtte        \\
             & Mobilitet              & Oppetidskrav                  \\
             & Billig løsning         & Utviklingstid prototype       \\
             & Monitorering           & Berøringsskjerm for Parkinson \\
             & Fjernstyring           & Potensielt daglig lading      \\
             & Sporing                & Lage enkelt nok grensesnitt   \\
             &                        & Driftsgaranti / seriøsitet /
    overlevelsesevne \\
    \\
    \textit{Eksternt}
             & \textbf{Muligheter}    & \textbf{Trusler}              \\
             & Høy årlig pris         & Mettet marked                 \\
             & Krever hussentral      & Direktiver og godkjenninger   \\
             & Sakte innovasjon       & Priskrig                      \\
             & Få GPS-produkter       & Lavt volum                    \\
             & Ingen skjerm           & Gode nok eksisterende produkt \\
             & <<Disruption>>         & Sensorintegrasjon sentral     \\
             &                        & Tregt salg, offentlig anskaffelse \\
             &                        & Store, seriøse aktører \\
             &                        & Tilbyr andre, relaterte produkter \\
             &                        & Patenter \\
             &                        & Smarttelefoner \\
  \end{tabular}
  \caption{SWOT-tabell for tynnklient trygghetsalarm.}
  \label{table.swot}
\end{table}

Med utgangspunkt i tabell \vref{table.swot} ...

Lokasjonsfunksjonalitet tilbys omtrent av halvparten av de produsentene vi har
sett på, og kun 35-40\%{} av kommunene tilbød dette i 2014
\cite{org.alarmmottak}.  Det har blitt utført studier om hvordan best utnytte
denne teknologien \cite{sintef.trygge.spor}, og vi vil anta at de fleste
kommunene fremover vil bytte til enheter med brukerlokalisering.

% Merk også, mange bruker alarmen bare når de skal ha hjelp til noe trivielt,
% feks å vaske opp søl på gulvet, og sånt. Vår sak MED SKJERM (har bare sett 1
% konkurrent som har det, MAMBO2) kan faktisk man skrive sånt, slk at
% sykepleiere kan prioritere. Kanskje også lurt om vi utvider på web-siden slik
% de kan gjøre mer ting, feks integrerer med eksisterende systemer,
% vaktsystemer osv. Ved hjelp, kult om en kan bare rope høyt og så varsler den.
% Perfekt hvis man ikke klarer å nå knappen men faller eller får
% epilepsianfall. Hadde også vært kult med hearbeat sensor eller noe sånt.

% Cite regler for offentlig anskaffelse:
% https://www.regjeringen.no/nb/dokumenter/veileder-offentlige-anskaffelser/id437022/
% Cite gjerne de som gikk konk (Merito) ifbm seriøsitet/overlevelsesdyktighet.
% TIpper kommunene ser mye på det. Tror også det er lett å bli bedre på
% funksjonalitet, men det er mange, så tror ikke det er det de primært ser
% etter. Det de trenger er jo bare den knappen.

% Generelt vil jeg frem til: Det vi må satse på er masseavtale, altså lage en
% avtale som leverer for et helt fylke eller kommune. Må utnytte partnere
% (domeneeksperter, SINTEF, feks). Høy konkurranse kan vi snu til en fordel med
% at vi henvender oss til de som sitter med kunnskap men ligger bak. De bør
% være kjempeinteressert i teknologi. Pris, med vår teknologi kan vi lage noe
% som gjør at vi har mye å gå på når det gjelder pris. Å havne i priskrig er
% ikke så lurt, men vi har mye å gå på. Derfor begynner vi med å si at
% enkeltpris er omtrent lik de andre, men hvis vi kjøper i volum så slår vi
% VELDIG mye av på årisen, bruker altså opp marginhøyden på dette.
% Offentlige anskafelser er vanskelig, men merk 110-Telemark, vi kan faktisk
% kanskje få til noe lignende. Vi kan jo vise til telemark, dette blir uansett
% bare en lokal strategi, funker ikke på nasjonalt plan og er et juridisk
% minefelt hvis vi gjør feil. Patenter kan også beskytte oss til en viss grad.
% Vi bør i hvertfall ha noen sentrale som kan brukes, selv om vi bare vil bruke
% de som beskyttelse, ikke offensivt (for det koster penger).
% At markedet er mettet, er vår løsning så mye bedre enn det vi har? Tror det
% går mye på faktisk oppsett og hva de opplever av oppetid.

% Grupper først per del og så skriv om gruppen.
% Hvordan styrke sterke sider:
% Hvordan redusere svakheter: Rundt halsen, testing, pilotprogram

% Hvordan utnytte muligheter: Grupper: Pris, teknologi, funksjonalitet
% Må vise at det funker med et pilotprogram og anbefalinger.
% Må vise at man kan spare penger på dette.

% Hvordan avverge trusler: Grupper: marked, regler, status quo
% Utnytt konkurranse ved å henvende til de som ligger
% på 2. plass. Offentlig anskaffelse er et problem, m

% TODO: Skriv tekst om hvordan gjøre det under.
% Hvordan:
% - Styrke sterke sider
% - Redusere svakheter
% - Utnytte muligheter
% - Avverge trusler


\section{Forretningsmodell}

\subsection{Kundesegment}

% • Hvem er bedriftens kunder ?
% • Hvilke behov har kundene ?
% • Hva vektlegger kundene ved valg av leverandør av denne type
% produkter/tjenester (kjøpskriterier) ?
% • Kommenter din evne til å innfri disse i forhold til konkurrentene ?

% TODO: Private kunder kan vi egentlig bare se vekk i fra i første omgang.
% Dersom vi ikke klarer å selge til kommunen så blir det heller ikke noe for
% private kunder.

% Kommuner er kundene våre, og i mindre grad privatpersoner.
% Kommunene har behov for en billigere løsning (vi ønsker alle å hjelpe folk,
% men de har ikke budsjett til å tilby alle), de ønsker mobil enhet med GPS for
% tracking, de ønsker enkel installasjon og oppsett, de ønsker
% fjernprogrammering, de har generelt behov for mye mer avanserte tjenester og
% mer funksjonalitet
%
% Privatpersoner har samme behovene, men spesielt tror vi at pris kan være
% behjelpelig for å øke salget av disse. Dette er folk som ikke har vanlig
% mobil eller som er for vanskelig å bruke eller ikke støtter ting som fall,
% inaktivitet og monitorering. Altså gamle som er for friske eller av andre
% grunner ikke kvalifiserer for subsidier gjennom kommune. Kan også gjelde
% yngre med feks funksjonshemminger. Målet er jo å få flere folk til å kunne bo
% hjemme selv.

% (1) Kommuner, gjennom Hjelpemiddelsentralen, (2) Gjennom partnersalg, (3)
% Privatpersoner --- rangert etter størrelse på marked

Primært eldre som får trygghetsalarm gjennom Hjelpemiddelsentralen, men også
privatpersoner som ikke er kvalifisert for offentlig hjelp. Dette kan være
personer som er engstelige eller personer med funksjonshemminger.

Eksisterende trygghetsalarmer mangler i stor grad tilkobling internett, de er
ikke mobile uten ekstrautstyr og mangler oftest GPS \cite{sverige.alarm}.
Det oppleves også det mye feil med alarmene og må kontrolleres manuelt. Dette
skjer i snitt én gang per døgn.

Da Norge er blant de fremste i verden på å ta i bruk velferdsteknologi
\cite{telenor.undersokelse} regner vi med at man har samme behov også i
utlandet.

Behovet er først og fremst å kontakte hjemmehjelpen. Dette kan gjelde akutte
medisinske situasjoner hvor hjelp er nødvendig, men også fall, innbrudd, brann
og lignende. Privatkunder har også behov for kontakt med medisinsk personell.
Dette tilbys gjennom privattjenester. Behovet er størst i hjemmet, men gjelder
også når man er ute av huset.

Den største brukergruppen får dette produktet gjennom Hjelpemiddelsentralen.
Derfor blir denne sentralen å anse som en kunde. De vil vektlegge kvalitet,
pris, funksjonalitet og brukervennlighet.

Vi kan tilby alarmer som er billigere i innkjøp --- man trenger ingen
hussentral, og enhetsproduksjonskosten er lavere. Videre kan vi tilpasse
funksjonalitet for enkeltbrukere, også etter enhetene er tatt i bruk, og
eventuelt justere månedsbeløpet etter dette. Vi kan tilby sentralene å spore
brukeren eller opprette kontakt eller sende varsler.  Nye tjenester kan
opprettes i samarbeid og tilbys uten å trenge erstatte enheten.

Privatkunder vil ha samme behovet som første gruppe, men gjerne med mer
funksjonalitet. For eksempel ønsker man kanskje å bli varslet dersom det er
glatt ute og så videre.

\subsection{Kundeverdi og verdiløfte}

% • Hva er bedriftens unike styrke i konkurransen om kundene?
% • Hva tilbyr bedriften sine kunder ?
% • Hvilke verdier skaper bedriften for sin kunder (verdiløftet) ?
% • Hvor stor antas betalingsvilligheten hos kundene å kunne være – og hva
% bestemmer denne?

Vår unike styrke er å levere en innovativ, billigere og mer funkjsonelt produkt
enn det som finnes på markedet fra før. Vi tilbyr en enhet som fungerer både
innendørs og utendørs.

Verdien vår er å gjøre brukeren mer mobil samtidig som kostnader reduseres.
Kanskje er det flere privatkunder som ønsker en slik løsning. Kan også
fjernstyre enheten. Dette kan være nyttig for å hente inn bruksmønstre,
monitorere om brukeren er oppe og går, men også fjernstøtte --- når man før
måtte gå med alarmen til en tekniker kan man nå hjelpe brukeren med feilsøking
over enheten selv, i stor grad.

Betalingsvilligheten er høy, da det er lovfestet at staten skal kunne tilby
brukerne denne type tjeneste. Privatkunder som føler sterkt behov for en alarm
vil også antas å betale det som er nødvendig. Blir bestemt ut fra,
selvfølgelig, produksjonskostnad, men også det faktum at dette er er relativt
lite marked. Nettverksoppkobling koster også penger, og sykepleiersentralene
tar også en stor del av månedsprisen.

\subsection{Distribusjonskanaler}

% • Hvor eller på hvilken måte leverer/distribuerer bedriften sine
% produkter/tjenester ? 
% • Hvilke kanaler brukes for å kommunisere/markedsføre kundeverdi og fortrinn ?

%%%%%%%
% MERK: Dette går mer på levering, distribusjon, ikke salg. 
% Kjøper hver kommune inn dette? -> er per fylke. De har sikkert kjempelyst til
% å spare penger.
%%%%%%%

Vi sender til hjelpemiddelsentralen, som da skal enkelt kunne sette opp
tjenesten selv. Da enhetene er koblet opp mot nett og har en skjerm kan de lett
settes opp til hver bruker på stedet. De kan verifisere med sentralen at dette
er satt opp riktig. Sentralen har et grensesnitt på nett for drift. På denne
måten skal det være mulig å kunne levere tjenesten på dagen. For privatkunder
gjelder det samme, men de setter den da opp selv. Sender da i posten.

Markedsføring skjer i første omgang direkte med Hjelpemiddelsentralen. Vi må ha
et møte med demonstrasjon og inngå avtale om et pilotprogram. Etter de første
enhetene er tatt i bruk skal vi profilere oss gjennom bransjeblad (som
Bransjeorganisasjonen for helse- og velferdsteknologi) og på konferanser.

Vi skal ha en sterk tilstedeværelse på nett, være lett å søke opp og gi en god
presentasjon. Til slutt kan vi ha reklame i ukebladet som Se og Hør, Allers og
så videre, for at potensielle sluttbrukere kan finne oss.

\subsection{Kunderelasjon}

% • Hvordan bygges kunderelasjoner?
% • Hvordan opprettholdes gode kunderelasjoner over tid?
% • Hvem er dine viktigste konkurrenter?
% • Når og hvorfor blir disse valgt?

% Hvem er egentlig kundene? Privatkundet har vi nevnt, og partnere, men virker
% som det egentlig er hjelpemiddelsentralen. Skal vi selge dette selv til
% hjelpemiddelsentralen? De vil nok ha flere tjenester enn den vi har, ting som
% kan kobles sammen.
% - Definitivt selge direkte til privatkunder
% - Definitivt selge ti hjelpemiddelsentralen, men de vil nok ha andre typer
%   sensorer også. Det gir mulighet for å lage flere produkter, gitt vi har
%   suksess for de andre. De har folk som kun har trygghetsalarmen og ikke noe
%   mer enn det. Blir fort utvikling av røykvarslere også.
% OK, ser ut som det blir to kunder: Privatpersoner og kommuner. Kommuner deler
% ut alarmene. Hvem betaler for dem? Kjøper disses inn per kommune?
% Er ihvertfall kommunene som tar regningen, enten delvis eller helt (feks
% Meldal).

I nesten alle tilfeller blir det snakk om partnerutvikling. Tror aldri vi
slipper unna dette, så det kan bli vanskelig å skalere så lenge vi er på
komponentnivå.

Dette er en bransje hvor det er vanskelig å skalere, for det krever mange folk.
Vi må dermed satse på at kunden i stor grad skal klare seg selv, men det skjer
kun på sikt.

Konkurrenter blir andre som er i relatert marked. Største konkurrent blir
giganter som Qualcomm. Fordelen vår er at vi har en skyplattform som Qualcomm
sannsynligvis aldri er interessert i å ha. Faktisk kan de bli en partner, der
vi bruker mye Qualcomm-komponenter i vår pakke, men med vår mer ferdige profil
og vår plattform som styrer disse.

Vår skyplattformen vil kundene våre ha programvare som kjører i skyen. Derfor
må vi her ha en enkel måte for self-service. A la google analytics, Amazon AWS
dashboard. Vi må ta månedlig avgift per device, eller per CPU og nettbruk.
Huff, er ikke helt dette jeg så for meg. Men er helt klart et marked for at
folk lager smart-appliances selv, og da gjennom oss, helt klart. Da blir jo
konkurrent fort Apple og Samsung, for de kan kobles opp til mange typer, de har
jo homekit og sånt og ser for seg å styre dette fra mobilen.
Men sånn sett så vil vi ligge godt under radaren lenge.

\subsection{Inntektsstrøm}

% • Hvordan skapes inntekter fra kjernevirksomheten?
% • Hvordan oppnås eventuelt andre inntekter?

Vi tar en gitt pris for selve utstyret og deretter månedlig betalt for drift.
Andre inntekter kan oppnås ved at vi utvider programvaren med ny
funksjonalitet. Med vårt system trengs ikke enhentene erstattes for å ta i bruk
ny funksjonalitet, det kan aktiveres ned på individuelt behov samme sekundet
som vi publiserer ny kode.  Eksempel på ny funksjonalitet er medisinalarm,
vekkerklokke, <<geofence>>, programvare som lærer daglige mønstre og sier fra
ved avvik (for eksempel at personen ikke står opp på normal tid).

På sikt bør vi satse på relaterte produkter, for eksempel sensorer som sjekker
om kjøleskap er åpnet, brannalarm, fallsensor og så videre. Dette for å styrke
forholdet til kunden vår, altså låse de inn med at de har flere produkter fra
oss.

På lang sikt kan vi bruke samme fysisk maskinvare til å nå andre nisjemarkeder.
Som nevnt, taksameter, overfallsalarm og med utvikling av større formfaktorer
kan vi tilby innbruddsalarm i hus og så videre.

\subsection{Nøkkelressurser og kritiske suksessfaktorer}

% • Hvilken gjennomføringsevne har bedriften?
% • Kompetanse og personlige egenskaper som gjør deg/dere spesielt egnet til å
% utvikle og drive bedriften?
% • Hvilke viktige ressurser og tilleggs-kompetanse trenger bedriften …
%    o på kort sikt?
%    o på lenger sikt?
% for å realisere forretningside og kundeverdien (“verdiløftet”)?

% TA MED: At vi blir sett på som seriøse, at vi har kraft og kapital og antall
% ansatte til å dra driften, kommunene satser nok vanskelig på de man anser som
% korttidslevede (flyktige). Tillitsbygging er viktig, men det kan vi bare
% gjøre over tid (fra bioenergi).

Vi brenner for vår langtidsvisjon --- å realisere virtuelle, tynne
mobilklienter.

Vi har lang erfaring med driftskritiske systemer på programvare. Trenger
maskinvarekompetanse.% Denne delen er vel litt utafor, skal vi virkelig ta med
% personlige egenskaper? Kan si vi brenner for dette for vi har en
% langtidsvisjon vi virkelig brenner for.

Vi trenger domeneeksperter på velferdsteknologi, spesifikt trygghetsalarm. Vi
trenger erfaring med offentlig innsalg og kvalifisering av elektroniske
konsumentprodukter til offentlige direktiver og regler. Når vi skal globalt må
vi ha rådgivere for hvert land vi tar fatt i. Det er store kulturforskjeller,
selv i Europa, og disse forsterkes med at det er et nisjemarked vi skal inn i.
Vi trenger hardware-eksperter (maskinvare).

% Kort sikt: Hardware-eksperter, offentlig salg, direktiver og regler,
% domeneeksperter
% Lang sikt: Global handelsrådgivere, økonomer for hvordan optimalisere
% inntjening når vi er igang, markedsføring for å vise oss igjen. I første
% omgang skal vi bare bevise at vi har livets rett, men snart må vi bevise at
% vi har rett til overleving. Trenger mobileksperter.

% Hva slags tilleggskompetanse trenger vi: 

% Hva er forskjellen på partnere og nøkkelressureser?
% Kritiske suksessfaktorer: Fungerende prototype, pilotprogram, første salg,
% første inntekt.
% Designere, må ha noen som kan lage dette

\subsection{Kjerneaktiviteter}

% • Hvilke kjerneaktiviteter må bedriften selv utføre?
% • Hvilke (kjerne-) aktiviteter kan/må settes ut til andre?
% • Hva er de mest kritiske faktorer for å lykkes mht lønnsom kommersialisering?

% --> Programmering, prototyping, utvikling av produkt 1. Selve assembly kan
%  gjøres i Kina. Design må vi få andre til å gjøre, vi har ikke kompetansen.
%  Daglig økonomi og regnskap kan vi gjøre slv helt helt i begynnelsen, bør
%  bruke de som har peiling når vi skalerer, for de har masse kule triks å
%  utnytte, har vi hørt.
% Kritiske faktorer: Kunde nr 1. Lønnsomhet, vi må ha en viss masse. Burde være
% mer spesifikke her. En faktor er båndbreddebruk og batteritid. Dersom det
% feiler her så feiler alt. Dvs batteri kan vi fikse for dette produktet med å
% ha rundt halsen. Direktiver, kan være ting her vi ikke er klar over, feks at
% en slk skjerm ikke kan brukes, kan være regler for denne brukergruppen. Bør
% ikke si for mye av at vi ikke vet ting, for det burde vi ha sjekka opp.

% Prototyping, utvkling av maskin- og programvare (alarm + plattform).
% Følge opp partnere i pilotprogram.
% Fokuserer kun på denne partneren før vi går ut for å finne nye, men vi må ha
% noen konkrete i sikte.
% Lengre sikt: Når vi tjener penger kan vi begynne å eksperimentere med
% generelle brukergrensesnitt som skal konkurrere med smartklokke.

\subsection{Partnere}

% • Hvilke partnere og leverandører samarbeider bedriften med for å kunne levere
% på sitt verdiløfte (2)?
% • Hvilke er avgjørende viktig på kort og på lenger sikt?

% Kanskje Lærdal Medical også på utvsiden? Men bør jo ikke ha for mange.

% Vedr domenekspert, finn de som ikke er ledende men ligger bak. De skriker
% etter fortrinn og vil sikkert lett være partner. Videre er det av X aktører
% veldig få som LAGER produktene selv, mange bare videreselger. Intersection av
% disse to er de partnerne vi er ute etter, feks (skriv inn navn). Tell X sånn
% ca.

\begin{itemize}
  \item En etablert aktør på markedet, for eksempel Dignio eller Numa Security.
  \item Telenor Objects, eid av Telenor og Den Norske Stat med formål å tilby
    totalløsninger for velferdsteknologi
  \item Hjelpemiddelsentralen, som også blir vårt største potensielle kunde
  \item Universiteter, for eksempel studentoppgaver i teknisk design eller
    hjemmepleie.
\end{itemize}

\subsection{Kostnadsstruktur}

% • Hvilke kostnader vil bedriften ha for å utvikle og drifte sitt
% forretningskonsept?

% Utvikling: prototyping, må kjøpe inn noe hardware, software har vi. Er mest
% snakk om tid.
% Drift: Servere, støttepersonell (vakttelefon). Må satse på at kommunene selv
% lett kan ordne ting, feks ved å bare ta med en ny device til brukeren og så
% aktivere på stedet (switche over). Kanskje vi kan ha en skikkelig ekte app
% (native) på iPhone/Android som sykepleiere har? Kanskje vi kan sjekke
% posisjonen til begge og route til de som er nærmest når alarmen går, sånne
% ting? <== God idé, kanskje sykepleierne selv kan ha en sånn device? Dobbelt
% salg! De ønsker kanskje ikke å installere apps på sin egen device heller.
% <============ GOD IDE!

\section{Vekstambisjoner og marked}

% Først trygghetsalarm, deretter ambisjoner om å selge disse internasjonalt.
% Deretter konkurrere med Apple og Samsung. Eventuelt kan vi på dette
% tidspunktet endre til å tilby komponenter og plattform til nisjeprodusenter
% globalt. Vi kan lage en markedsplass for programvaren og gjøre det mulig for
% tredjeparter å tilby halvfabrikat (komponenter og programvare, feks GPS +
% programvare for å aksessere den, gitt at vi ikke går etter generell
% mobiltelefon).

\subsection{Vekstmål}

\subsection{Analyse av produkt og marked}

\section{Økonomiske forhold}

\subsection{Produktkalkule}

\subsection{Salgsbudsjett}

\subsection{Driftsbudsjett}

% Trenger vi kontor og sånt? IKke når vi lager prototype, evet bruke mess &
% order. Dette kan vi gjøre hjemme LENGE. Det billigste er kanskje å leie en
% vanlig leilighet eller noe sånt, billig, hvis vi virkelig trenger lokale. Når
% vi skal ha møter må vi jo uppe, men kan bruke hotell og sånt om nødvendig.
% Kan bruke egne maskiner, men må kjøpe litt hardware. Koster minimalt, samme
% gjør AWS som vi kan kjøre fra egne pcer. Dette var prototyping. Når vi skal
% opp må vi si hva vi trenger baseline for å ha drift. Må da ha kontor, noen
% tusenlapper i mnd på AWS, osv.

% Lån, kassekreditt? spørs om vi får det hos noen lev. trenger det egentlig
% ikke heller. Mye bra her egentlig:
% http://distriktssenteret.no/wp-content/uploads/2013/03/abp-bioenergi-as-forretningsplan-pelletssalsselskap.pdf
% Men IKKE stjel noe.. bare generelle idéer.

\subsection{Likviditetsbudsjett}

\subsection{Investeringsbudsjett og finansiering}

\subsection{Oppsummering og handlingsplan}

\chapter{Analyse og diskusjon}

\section{Prosessen}
\label{prosessen}

Vi startet med en visjon om en helt ny type mobiltelefon, basert på
tynnklientkonseptet. Dette er presentert i \textit{prosjektskissen}. Når vi
imidlertid undersøkte dette i detalj fant vi ut at gjennomsnittlig datamengde
ville ligge på rundt 9-10 Gb per måned. Dette er langt over det som er vanlig
for både forbrukere og teleoperatører i dag, og derfor er det uoverkommelig
dyrt. Vi var også usikre på hvor mye strøm en tynnklient ville trekke på grunn
av økt trådløskommunikasjon.

Deretter begynte vi å se på en ny produktkategori: Smartklokker. Hovedproblemet
her er å konkurrere med giganter som Apple og Samsung, og vi er forøvrig for
sent ute til å kunne være med å forme forbrukeres forventninger til denne type
produkt.

På dette tidspunktet begynte vi å konkretisere produktet mer. Tynnklienter
krever en døgnkontinuerlig tjenerpark, dermed må vi ta en månedsavgift for å
dekke driftsutgifter. Dessuten ville det være lurt å lage et relativt enkelt
produkt, sammenlignet med en smartklokke. Vi undersøkte bruksområder for hvilke
typer applikasjoner som behøver konstant tilgang på nett, nisjespesialisering og som passer inn i
abonnementsmodellen. Vi hadde flere alternativer, men trygghetsalarmen så ut
til å ha et skrikende behov for en ny vri. Dette gjorde arbeidet mye lettere.

I læreboken står det faktisk om søken av .... blabla .... TODO

TODO: Si litt om disse stadiene for å innovere. Og nevn at vi kombinerer
faktisk to kategorier (tynnklient/virtualisering vs mobilenhet).

Vi har forøvrig stor tro på vårt konsept, for en kan bruke samme maskinvare og
plattform til en rekke andre nisjeprodukter: <<Hands-free>> systemer for
industriarbeider, inkludert talestyrt plukking på Bama, styring og panel for
innbruddsalarmer i hus, taxi taksameter, grensesnitt på brusautomater og
lignende --- til og med grensesnitt på kassaapparat.  Hver produkttype krever
kun ny programvare på tjenersiden, og man kan oppnå <<economy of scale>> ved å
bruke samme maskinvare på en rekke områder.

Imidlertid er det en ulempe med dette, og det er at hver kategori krever
domeneekspertise. For å være seriøse må vi allokere mange folk per type produkt
og bygge opp kunnskap. Det er også nye separate markeder som vi er nødt å
tvinge oss inn i hver gang.


\section{Marked og utvikling}

I figur \vref{fig.mottakere} ser vi at antall mottakere av trygghetsalarmer har
holdt seg mellom 73--74 tusen de siste årene.  Prisen på en alarm er rundt seks
tusen kroner i året, eller fem hundre i måneden. Dette inkluderer utstyr,
eventuell mobiloppkobling og vaktsentral. Det er vanlig at kommunene
subsidierer dette kraftig. Dermed kan vi anslå at markedet i Norge rundt 438 millioner
kroner per år.

Det finnes også rundt 20 tusen mottakere som bruker alarmen kun deler av året.
Det betyr at markedet sannsynligvis er noe høyere, men betyr også at det bør
være behov for enkel installasjon og oppsett av enhetene.

\begin{figure}
  \includegraphics[width=\textwidth]{plots/alarmer-land.pdf}
  \caption{Mottakere av trygghetsalarmer i Norge.
    Tallgrunnlag fra Statistisk Sentralbyrå \cite{iplos.2013}.}
  \label{fig.mottakere}
\end{figure}

Med andre ord bør det være stor mulighet for å introdusere et billigere
produkt, selv om det bare er en del av hele regnestykket.

Når det gjelder utvikling ser vi av figur \vref{fig.mottakere} at mottakere av
generelle hjemmetjenester øker blant de som er yngre enn 67 år. Trygghetsalarm
er bare en liten del av dette, men vi ser også at den yngre brukergruppen øker
noe. Dette er en yngre brukergruppe som kanskje er mer teknologivandte. De vil
da sannsynligvis ønske flere funksjoner.

% Ta med notater fra alt annet, spesielt om "viability", altså hvor realistisk
% det er, farer, osv. MÅ også ta med ting fra pensum!

% Husk "planned obsolesence", "minimum viable product", evt "exit-strategier"
% (mer osm i pivot-alternativ).

% Skim også gjennom boken

\section{Konkurranse}

Med et marked på rundt 480 millioner kr i Norge, er det likevel svært mange
konkurrenter. De fleste har ikke så avansert funksjonalitet, men grunnproduktet
er enkelt å utvikle, og vi anser derfor det er så mange konkurrenter.

Kommunene er interessert i aktører som er seriøse, og gjerne tilbyr flere
tjenester. For eksempel ble Aleris nylig gitt kontrakt om å levere
trygghetsalarm i Oslo \cite{telenor.aleris} til vel en tredel av brukerne.
Dette er en stor bedrift med mange ansatte og tilleggstjenester. De er seriøse
og satse på velferdsteknologi.

Effektivisering av tjenester og besparelse av penger står også svært sentralt
for mange kommuner. Derfor har man i Telemark fylke samlet
brannvarslingstjenestene til én sentral i Skien. Alle som vil kobles opp mot
automatisk brannvarsling må gjøre dette gjennom 110-Telemark. Dette er en
datterorganisasjon av kommunene. Samtidig tilbyr de trygghetsalarm gratis på
toppen av brannvarslingstjenesten. Dette stenger ute konkurrenter. I brev til
Konkurransetilsynet \cite{telemark.konkurransetilsyn} ble dette påklaget av
Hjelp24, eid av Gjensidige og nå del av Stamina Group, men ble til slutt
avslått. Begrunnelsen var at det ikke var snakk om <<utilbørlig utnyttelse>>,
fordi dette gjelder kun 18 av 431 kommuner i det nasjonale markedet.

Det kan med andre ord være svært vanskelig å komme inn på markedet. Det beste
er nok å inngå avtale om pilotprosjekt med en kommune, eller et helt fylke. Ved
å tilby tjenesten for et helt fylke kan man tilby effektivisering for flere
kommuner. Dette er nok en bedre strategi enn å se på enhetskostnaden. Men det
krever en aktør av en viss størrelse for å bli tatt seriøs.

