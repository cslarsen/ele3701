% Norwegian
\usepackage[norsk]{babel}

% Hyperlenking
\usepackage[hyphens]{url}

% Latin1 Times Roman font for Norwegian
\usepackage{times}
\usepackage[T1]{fontenc}

% Endre "kapittel X" til noe annet, gitt vi bruker Babel
\addto\captionsnorsk{\renewcommand{\chaptername}{Del}}

% Bruk romerske tall for kapitlene
% TODO: Fiks dette

% Be able to change the title style
\usepackage{titling}

% Easier references
\usepackage{varioref}

% Bibliography in table of contents
\usepackage[nottoc]{tocbibind} 

% Norwegian 1.5 linespacing
\usepackage{setspace}

% BI guidelines:
\onehalfspacing
\usepackage[top=2cm,bottom=2cm,left=5cm,right=2cm]{geometry}
% Chicago style (TODO: Funker ikke på min versjon av latex)
%\usepackage[authordate-trad,backend=biber]{biblatex-chicago}

% Enable UTF-8 text
\usepackage[utf8]{inputenc} % enable UTF-8 text

% Adds bibliography to table of contents
\usepackage[nottoc]{tocbibind}

% Use newlines in paragraph instead of indentation
\usepackage{parskip}

% Build index, using the new imakeidx in TeX Live, see
% http://tex.blogoverflow.com/2012/09/dont-forget-to-run-makeindex/
%\usepackage{imakeidx}
%\makeindex

% Uncomment this if you want to show all \index{} locations in the document
% in the margins (good for proofreading)
%\usepackage{showidx}

% Hyperlinks, clickable references
% Hidelinks turns off colored box on some PDF viewers
\usepackage[hidelinks]{hyperref}
%\usepackage{hyperref}

% On Mac OSX, in ViM, when I type ALT+SHIFT+SPACE (often typed accidentally
% after writing {\some-command ...}, then LaTeX may give me the error
% "Unicode char \u8:  not set up for use with LaTeX.".  This character
% in UTF-8 actually means what the tilde (~) means in LaTeX, but *I*
% usually mean just a space.  This command replaces those with a
% normal space:
\DeclareUnicodeCharacter{00A0}{ }
% If I really want the tilde-behaviour, I can use the command
%\DeclareUnicodeCharacter{00A0}{~}
%
% The above solution was taken from:
% http://tex.stackexchange.com/questions/83440/inputenc-error-unicode-char-u8-not-set-up-for-use-with-latex

\usepackage{amsmath}
%\usepackage{xfrac}

% URLs in citations look terrible, so fix that
\urlstyle{same}
\hypersetup{breaklinks=true}
\Urlmuskip=0mu plus 1mu\relax
\usepackage{ragged2e}

% To include entire PDFs
\usepackage{pdfpages}
