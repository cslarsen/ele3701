\section*{Innledning}

Vi har valgt å skrive en forretningsplan for et produkt som per dags dato ikke
finnes på markedet, men som vi tror har svært stort potensial.

Etter vi har forklart forretningskonsepter vil vi presentere hvordan vi akter å
lage en mest mulig realistisk plan for et slikt produkt.

\section*{Forretningsidéen}

Alle smarttelefoner på markedet i dag er såkalte \textit{tykke klienter}. Det
betyr at de utfører alle oppgaver lokalt på selve telefonen fysiske maskinvare.
Dette krever mange kraftige komponenter, og leder til høyt strømforbruk og dyr
produksjonskostnad.

Vår idé er å lage en \textit{tynnklientmobil}, der hele mobiltelefonen
virtualiseres og blir et program som kjører på en tjener i skyen. Den fysiske
mobilen blir da en dum terminal som kun mottar og sender bilde, video, lyd og
brukerinteraksjon som befølingstrykk.

Fordelene er mange: Telefonen vil bestå av minimalt med komponenter og bør
derfor bli svært billig å produsere per enhet. Dette leder til at telefonene
vil kunne bli tynnere --- i \textit{prinsippet} trenger man kun en skjerm og et
GSM chipset \footnote{Man vil fremdeles trenge noen flere komponenter,
  inkludert en enkel CPU. Men langt billigere og færre komponenter enn en
smarttelefon.} --- og få dramatisk bedre batterilevetid.

Ikke minst blir forholdet mellom produsent og kunde mye tettere; kundene må
abonnere for å kjøre mobiltelefonen i skyen. Dette skaper grobunn for en del
interessante andre retninger.

\section*{Forretningsplan}

I denne skissen har vi valgt å bruke <<the business model canvas>>
\cite{oswalder} som utgangspunkt for en full forretningsmodell.

Den består av følgende punkter:

\begin{itemize}
  \item Nøkkelpartnere
  \item Nøkkelaktiviteter
  \item Nøkkelressurser
  \item Verdiforslag (<<value propositions>>)
  \item Kundeforhold
  \item Kundesegment
  \item Salgskanaler
  \item Kostnadsstruktur
  \item Inntektsstrømmer
\end{itemize}

\section*{Utførelse}


% Datainnsamling
% - Markedsdata (markedsstørrelse, low-mid end)
% - Demografiske data, inntektsgrupper osv, spredning
% - Priser for deler og komponenter, priser for oppskalert produksjon
% - Ha et realistisk forhold til produksjonskostnader
% - Utviklingskostnader, må da se på hvilke hovedfaser som må gjøres
% Kontakte feks Telenor Group Innovasjon
% Kontakte Innovasjon Norge

\cite{bessant}.
