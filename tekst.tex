\section{Innledning}

Vi har valgt å skrive en forretningsplan for et produkt som ikke enda
eksisterer.

Etter å ha presentert konseptet vil vi ta for oss de forskjellige momentene i
\textit{<<the business model canvas>>}-modellen \cite{osterwalder}, som da
danner et grunnlag for videre jobbing mot den endelige rapporten.

\section{Forretningsidé}

Alle smarttelefoner på markedet i dag er \textit{tykke klienter}. Det betyr at
de utfører alle oppgaver fysisk på maskinvaren. Dette krever kraftige
komponenter som er dyre å produsere og som trekker mye strøm.

Vår idé er å utvikle en \textit{tynnklientmobil}, hvor den tradisjonelle
smarttelefonen \textit{virtualiseres i sin helhet} og kjører på en tjener i
skyen. Man sitter da igjen med et syltynt grensesnitt --- en dum terminal ---
som videreformidler lyd, bilde og berøringstrykk
mellom tjener og sluttbrukeren. Dette leder
til at

\begin{itemize}
  \item telefonen blir dramatisk billigere å produsere, da den krever færre og
    simplere komponenter,
  \item batteritiden vil øke betraktelig,
  \item kunden går fra å være en produkteier til å bli en tjenesteabonnent.
\end{itemize}

Det siste punktet er interessant fordi det kan muligens gjøre det mulig å
subsidiere produktet kraftig og ta fortjeneste på abonnementsdelen.

Vi tror at verden er på vei mot tynne mobile klienter. Vi ser allerede
tendensen, hvorpå noen produsenter som Mozilla med sitt mobile operativsystem
tilbyr en mer hybrid tykk- og tynnklientløsning.

\section{Kundesegment}

Vi må finne ut hvilke kundesegment vil skal satse på. Dette produktet har noen
kompromisser sammenlignet med eksisterende <<high-end>> mobiltelefoner, og det
er urealistisk å konkurrere med navn som Apple og Samsung.

Derfor bør vi sannsynligvis satse på et <<low>> til <<mid-end>> segment av
markedet --- personer som ikke må ha det nyeste og dyreste, men de som vil ha
et praktisk og billig alternativ.  Vi må kartlegge hva slags demografisk profil
ser for oss.

Samtidig må vi undersøke muligheter for å markedsføre mobilen i deler av
hverden hvor man ikke har råd til dyre mobiler. Feks i land som India, Pakistan
og Kina er det stor forskjell på rik og fattig, og det er mange fattige. Dermed
kan vi kanskje klare å få til en prissammensetning som 


en praktisk. Vi må drive markedsundersøkelse for å finne ut hva slags kundemasse
vi kan nå, sannsynligvis de 
at

har ikke lenger noe å si (\textit{alt} ligger på tjeneren til enhver tid),
oppgra

Det finnes også flere fordeler for kunden her, som vi nevner kort: Tyveri og
tap av mobil har ikke lenger noe å si, da \textit{alt} ligger på tjeneren til
enhver tid likevel,

Telefontyveri vil være verdiløst, data vil alltid ligge på se
Telefontyveri blir historie, da den fysiske mobilen kun er et vindu inn til 

\textit{i sin helhet}
telefonens oppga
kostba
krever kostbare komponenter som CPU, korttids- og langtidsminne, batterikomponenter som er kostbare å produsere og trekker mye strøm.

de utfører alle oppgaver lokalt, fysisk på selve telef
Alle smarttelefoner på markedet i dag er såkalte \textit{tykke klienter}. Det
betyr at de utfører alle oppgaver lokalt på selve telefonen fysiske maskinvare.
Dette krever mange kraftige komponenter, og leder til høyt strømforbruk og dyr
produksjonskostnad.

Vår idé er å lage en \textit{tynnklientmobil}, der hele mobiltelefonen
virtualiseres og blir et program som kjører på en tjener i skyen. Den fysiske
mobilen blir da en dum terminal som kun mottar og sender bilde, video, lyd og
brukerinteraksjon som befølingstrykk.

Fordelene er mange: Telefonen vil bestå av minimalt med komponenter og bør
derfor bli svært billig å produsere per enhet. Dette leder til at telefonene
vil kunne bli tynnere --- i \textit{prinsippet} trenger man kun en skjerm og et
GSM chipset \footnote{Man vil fremdeles trenge noen flere komponenter,
  inkludert en enkel CPU. Men langt billigere og færre komponenter enn en
smarttelefon.} --- og få dramatisk bedre batterilevetid.

Ikke minst blir forholdet mellom produsent og kunde mye tettere; kundene må
abonnere for å kjøre mobiltelefonen i skyen. Dette skaper grobunn for en del
interessante andre retninger.

\section*{Forretningsplan}

I denne skissen har vi valgt å bruke <<the business model canvas>>
\cite{oswalder} som utgangspunkt for en full forretningsmodell.

Den består av følgende punkter:

\begin{itemize}
  \item Nøkkelpartnere
  \item Nøkkelaktiviteter
  \item Nøkkelressurser
  \item Verdiforslag (<<value propositions>>)
  \item Kundeforhold
  \item Kundesegment
  \item Salgskanaler
  \item Kostnadsstruktur
  \item Inntektsstrømmer
\end{itemize}

\section*{Utførelse}


% Datainnsamling
% - Markedsdata (markedsstørrelse, low-mid end)
% - Demografiske data, inntektsgrupper osv, spredning
% - Priser for deler og komponenter, priser for oppskalert produksjon
% - Ha et realistisk forhold til produksjonskostnader
% - Utviklingskostnader, må da se på hvilke hovedfaser som må gjøres
% Kontakte feks Telenor Group Innovasjon
% Kontakte Innovasjon Norge

\cite{bessant}.
