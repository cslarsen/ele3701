\section{Innledning}

Vi har valgt å skrive en forretningsplan for et hypotetisk, men realistisk
produkt.

Etter å ha presentert konseptet vil vi si litt om hvordan vi planlegger å
utføre oppgaven. Etter dette kommer et grovt utkast til en forretningsmodell.

\section{Forretningsidé}

Alle smarttelefoner på markedet i dag er \textit{tykke klienter}. Det betyr at
de utfører alle oppgaver fysisk på maskinvaren. Dette krever kraftige
komponenter som er dyre å produsere og som trekker mye strøm.

Vår idé er å utvikle en \textit{tynnklientmobil}, hvor den tradisjonelle
smarttelefonen \textit{virtualiseres i sin helhet} og kjører på en tjener i
skyen. Man sitter da igjen med et syltynt grensesnitt --- en dum terminal ---
som videreformidler lyd, bilde og berøringstrykk mellom tjener og bruker. Dette
leder til at

\begin{itemize}
  \item telefonen blir dramatisk billigere å produsere, da den krever færre og
    simplere komponenter,
  \item batteritiden vil øke betraktelig,
  \item lagringsplass blir dynamisk,
  \item kunden går fra å eie et produkt til å bli en tjenesteabonnent.
\end{itemize}

Planen er å subsidiere selve enheten kraftig, men tjene på en noe dyrere
abonnementstjeneste.

Apple og Samsung jobber hardt for å levere mobiltelefoner som er tynnere,
billigere, har lengre batteritid og større lagringsplass, men hver generasjon
er bare inkrementelle forbedringer fra den forrige.  Med vårt konsept gjør man
et aldri så lite kvantesprang på alle disse punktene.

\section{Utførelse}

\subsection{Investeringsbehov og kostnader}

Det viktigste er å undersøke om idéen er økonomisk forsvarlig: Er det mulig å
tjene penger på dette produktet?

Vi må starte med å kartlegge hvor store investeringskostnader som trengs for å
bring produktet på markedet.  Dette inkluderer å

\begin{itemize}
  \item hente inn priser på komponenter på forskjellige ordrevolum
  \item estimere datasenterkostnader
  \item estimere utviklingskostnader for mobilenhet
  \item estimere utviklingskostnader for tjenerprogramvare
  \item undersøke hvilke godkjenninger som trengs for å selge produktet per
    region
\end{itemize}

Ved å undersøke \textit{commercial off-the-shelf}-tilbud kan vi beregne priser
for komponenter og programvare.  Når det gjelder utviklingskost og -tid bør vi
se etter tidligere, lignende teknologiprosjekter og lære så mye vi kan fra det.

\subsection{Markedsanalyse}

Vi må forstå hva slags segment av markedet vi ønsker å angripe. Det er svært
mange aktører på markedet, og konkurransen er knallhard. Derfor må vi ha en
smart og litt unik angrepsplan.  Planen er som sagt å kunne tilby en bra
mobiltelefon til de som ikke er villige til, eller har råd til, å betale for de
dyreste telefonene.  Samtidig styrer vi unna den laveste delen av markedet, da
dette allerede har bra og billige produkter fra produsenter som Nokia.

Her kan vi lete etter tall fra statistiske byrå og se etter forretnings- og
forskningsartikler i forskjellige tidsskrift.


\section{Forretningsplan}

For å lage et utkast for en forretningsplan har vi brukt momenter fra
\textit{the Business Model Canvas} \cite{osterwalder}. Dette skal danne et
grunnlag for videre jobbing med den endelige rapporten.

\subsection{Kundesegment}

I utgangspunktet er dette et produkt og en tjeneste for alle som ønsker en
smarttelefon.

Det er imidlertid urealistisk å konkurrere med de store, etablerte <<high-end>>
produsentene.  Vi bør dermed legge oss på et <<mid->> til <<high-end>> sjikt,
med mål å bevege oss oppover over tid.

Det er vanskelig å lage en klar demografisk profil basert på disse to
gruppene, men med litt markedsinformasjon kan vi klare å danne oss et mer
spesifikt bilde på hvilke segment vi ønsker å fri til.

Samtidig må vi huske på at veldig store deler av verden har ikke råd til dyre
mobiltelefoner. I land som India, Kina, Pakistan og de store urbaniserte
områdene i Afrika fins det mange folk som ønsker en billig men god mobil. Vi
ser for oss store kundemasser i storbyer, hvor det forøvrig er bra
mobildekning, som kreves for denne type produkt.

For å kunne tilby et økosystem rundt mobiltelefonen må vi få utviklere på
plass. Disse blir ikke kunder, men en gruppe vi skaper verdi for: De kan selge
og tjene penger på vår plattform.

Våre segment vil dermed bestå av

\begin{itemize}
  \item de som ønsker en billig mobil (vestlige land)
  \item de som ønsker en billig mobil (Øst-Europa, India, Pakistan, China)
  \item <<early adopters>> som ønsker å prøve ut en radikalt annerledes
    tjeneste (<<mid>> til <<high-end>> brukere)
  \item app-utviklere
\end{itemize}

\subsection{Verdiløfte}

Det verdifulle vi kan tilby våre kundesegment er

\begin{itemize}
  \item En helt ny type mobiltelefon, hvor produkt og tjeneste er ett
  \item Sikkerhet (vanskeligere at data kommer på avveie)
  \item Backup (alt ligger på skyen)
  \item Tyverisikkert (må være abonnent for å bruke mobilenhet)
  \item Ekstremt billig introduksjonskost, forskjellige abonnementsnivå og
    -priser (differensiere på lagringsplass brukt, osv.)
  \item Økt batteritid
  \item Tap av mobil er ikke et problem; en ny er svært billig og alt innholder
    lever likevel i skyen, så man kan fortsette der man slapp ved å gå ned på
    Narvesen, kjøpe en ny enhet og logge på
\end{itemize}

\subsection{Kanaler}

Vi ser for oss å selge selve enhetene i tradisjonelle butikker
(elektronikk-kjeder, mobilbutikker), men også i små kiosker (Narvesen, kiosker
på gaten i storbyer). Dersom vi kan tilby integrerte SIM--kort kan telefonen
aktiveres på stedet\footnote{Dette er litt utenfor oppgaveomfanget, men en
svært fristende mulighet: Å kunne tilby både produkt sammen med et
mobilabonnement.}.

For å øke bevissthet må vi drive massiv markedsføring. Dette bør gjøres noe
utradisjonelt, dog. For eksempel, i store, tett befolkede byer, ser vi for oss
kampanjer på gata. Med <<ready-to-go>> mobilavtale kan vi selge enhetene på
gaten.

Kundestøtte skjer gjennom apps i selve mobilenheten, eller på nettsider.

\subsection{Kundeforhold}

I stor grad ønsker vi å kunne tilby selvbetjening for kunder, hvor en kan gjøre
endringer på abonnement, sende inn feilmeldinger og få hjelp.

Utviklere må enkelt kunne distribuere sine apper på plattformen vår.  Det alle
de store produsentene gjør er å godkjenne hver app. Dette krever en stor
investering for å få til.  I begynnelsen kan vi tilby kun vår egne apps, før vi
åpner for eksterne utviklere.

\subsection{Fordeler}

Ved å lansere vårt produkt får vi en <<first-mover>>-fordel \cite{bessant}.
Imidlertid er det stor sjanse for at etablerte produsenter kan kopiere vår
modell med en med hybrid løsning, hvor noen tjenester kjører i skyen. Derfor må
vi satse stort på en kompromissløs tynnklientløsning.

Merk at mange av komponentene som behøvs for å lage telefonen er <<off the
shelf>>-varer, jmf.~Qualcomm SnapDragon chipset, Mirasol e-ink display, m.m.

Det blir imidlertid en stor investering på tjenersiden: Mens tradisjonelle
tynnklientløsninger bygger på Pareto-prinsippet --- for eksempel at når Gmail
tilbød 1 Gb lagringsplass ved introduksjon, så visste de at svært få kom til å
bruke opp alt --- så bruker folk mobiltelefonen svært mye mer. Derfor må vi
bygge en enormt avansert skytjeneste. Ikke minst, for å redusere latens
(<<latency>>) må vi plassere tjenere nære kundene, i en CDN--løsning. Dette er
ting som krever lang tid for å få opp, spesielt med tanke på all programvaren
som må spesialutvikles.

\subsection{Inntektsstrømmer}

Vi ønsker å gjøre det enkelt og billig å bli kunde. Derfor må vi subsidiere
mobiltelefonen kraftig og tjene inn dette på abonnementstjenesten.

\paragraph{Faste månedsbeløp}
Det mest grunnleggende abonnementet bør dekke et typisk månedsforbruk med data.

\paragraph{Betal for bruk}
Samtidig må det være mulig å betale mer for krevende brukere. For eksempel, de
som bruker mye lagringsplass kan betale per brukte gigabyte.

\paragraph{Andre muligheter}
Til slutt fins det en mulighet for å selge brukerdata. Dette er et svært
kontroversielt tema, spesielt i Europa, og kan være skadende på denne type
tjeneste hvor kunden har \textit{alle} sine data og apps i vår sky.  

Imidlertid bør en ha en mulighet for anonymisert samling av bruksmønstre. Dette
for å kunne forbedre automatiserte tjenester.

\subsection{Nøkkelressurser}

Det vil sannsynligvis være svært viktig å sikre seg patentbeskyttelse på disse
idéene tidlig. Ikke bare for konkurransebeskyttelse, men også for å kunne
stille opp med noe spesifikt i tilfelle vi blir saksøkt av andre. Å være aktør
på det globale mobilmarkedet er som å spasere på et minefelt av patenter.

\subsection{Nøkkelaktiviteter}

Før man kan selge den første mobiltelefonen kreves det enorme investeringer i
produktutvikling, etablering av infrastruktur for tjenere.

Parallelt med dette må man skape en effektiv infrastruktur for administrasjon,
salg, kundestøtte og lignende.

\subsection{Nøkkelpartnere}

Vi nevner kort

\begin{itemize}
  \item Mobiloperatører
  \item Komponentprodusenter
  \item Patentbyrå
\end{itemize}

\subsection{Kostnadsstruktur}

De viktigste kostnadene i første fase blir

\begin{itemize}
  \item Forskning og utvikling
  \item Etablering av datasentre
\end{itemize}


\subsection{Exit--strategy}

Selv om dette ikke er en del av <<the Business Model Canvas>> så vil vi nevne
kort muligheter for en <<exit-strategi>>.

De store produsentene vil nok sakte bevege seg i retning av tynne klienter, men
i en mer hybrid modell.  Derfor er en del av planen vår å være pionérer og
beskytte intellektuelt eierskap med patenter.  En \textit{exit-strategi} vil
dermed være å tilby kompetanse og IP til slike firma.
