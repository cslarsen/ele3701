\section{Innledning}

Vi har valgt å skrive en forretningsplan for et mobilprodukt og -tjeneste.

I denne prosjektskissen vil vi presentere konseptet og si litt om hvordan vi
planlegger å utføre videre arbeid. Deretter følger et grovt utkast til en
forretningsplan.

\section{Forretningsidé}

Alle smarttelefoner på markedet i dag er \textit{tykke klienter}. Det betyr at
de utfører alle oppgaver fysisk på maskinvaren. Dette krever kraftige
komponenter som er dyre å produsere og trekker mye strøm.

Vår idé er å utvikle en \textit{tynnklientmobil}, hvor den tradisjonelle
smarttelefonen \textit{virtualiseres} i sin helhet og kjører på en tjener i
nettverksskyen. Man sitter da igjen med et syltynt grensesnitt --- en dum terminal ---
som videreformidler lyd, bilde og berøringstrykk mellom tjener og bruker. Dette
leder til at
%
\begin{itemize}
  \item telefonen blir dramatisk billigere å produsere, da den krever færre og
    simplere komponenter,
  \item batteritiden vil øke betraktelig,
  \item ting som lagringsplass og prosessorkraft blir løsrivet fra fysiske
    komponenter, og
  \item kunden går fra å eie et produkt til å bli en tjenesteabonnent.
\end{itemize}

Planen er å subsidiere selve enheten kraftig, og tjene dette inn på en noe
dyrere abonnementstjeneste.

Apple og Samsung jobber hardt for å levere mobiltelefoner som er tynnere,
billigere, har lengre batteritid og større lagringsplass, men hver generasjon
er bare inkrementelle forbedringer fra den forrige.  Med vårt konsept gjør man
et kvantesprang på alle disse punktene.
\clearpage
\section{Utførelse}

Vi går her gjennom hovedmomenter for å jobbe videre med oppgaven. Når vi har
mer informasjon bør vi lage en \textit{SWOT-analyse} \cite{bessant}.

\subsection*{Investeringsbehov og kostnader}

Det viktigste er å undersøke om idéen er økonomisk forsvarlig: Er det mulig å
tjene penger på dette produktet?

Vi må starte med å kartlegge hvor store investeringskostnader som trengs for å
utvikle produktet og få det på markedet.  Dette inkluderer å
%
\begin{itemize}
  \item hente inn komponentpriser på forskjellige ordrevolum
  \item estimere datasenterkostnader
  \item estimere utviklingskostnader for mobilenhet
  \item estimere utviklingskostnader for tjenerprogramvare
  \item etablere en infrastruktur for salg og administrasjon
  \item undersøke hvilke godkjenninger som trengs for å selge produktet per
    region
\end{itemize}

Ved å undersøke \textit{commercial off-the-shelf}-komponenter kan vi beregne
priser på maskin- og programvare.  Når det gjelder utviklingskost og -tid bør
vi se etter tidligere, lignende teknologiprosjekter og lære så mye vi kan fra
det.

\subsection*{Markedsanalyse}

Vi må forstå hva slags segment av markedet vi ønsker å angripe. Det er svært
mange aktører på markedet, og konkurransen er knallhard. Derfor må vi ha en
smart og unik angrepsplan.  For å gjøre dette må vi hente inn
markedsinformasjon og informasjon om kundemassene. Vår plan er å tilby en noe
billigere mobil som ligger i sjiktet rett under <<high-end>>-markedet, med
potensiale å bevege oss oppover over tid.

For å gjøre dette kan vi lete etter tall fra statistiske sentre og analysebyrå
og se etter
forretningsanalyser og forskningsartikler i forskjellige tidskrift.

Det kan være lurt å kontakte Innovasjonsavdelingen i Telenor Group og
Innovasjon Norge for å hente inn slik informasjon.

\section{Forretningsplan}

For å lage et utkast for en forretningsplan har vi brukt momenter fra
\textit{the Business Model Canvas} \cite{osterwalder} og en forretningsplanmal
fra Innovasjon Norge \cite{innovasjon.norge}. 

\subsection*{Kundesegment}

I utgangspunktet er dette et produkt og en tjeneste for alle som ønsker en
smarttelefon.

Det er imidlertid urealistisk å konkurrere med de store, etablerte <<high-end>>
produsentene.  Vi bør dermed legge oss på et <<mid->> til <<high-end>> sjikt,
med mål å bevege oss oppover over tid.

Det er vanskelig å lage en klar demografisk profil basert på disse to
sjiktene, men med litt markedsinformasjon kan vi klare å danne oss et mer
spesifikt bilde på hvilke kundesegment vi ønsker å fri til.

Samtidig må vi huske på at mange ikke har råd til dyre telefoner. I land som
India, Kina, Pakistan og enkelte storbyer i Afrika finnes det et marked for
billige smarttelefoner. Storbyer med tett befolkning og god mobildekning.

For å kunne tilby et økosystem rundt mobiltelefonen må tiltrekke eksterne
utviklere.  Våre segment vil dermed bestå av
%
\begin{itemize}
  \item de som ønsker en billig mobil (vestlige land)
  \item de som ønsker en billig mobil (Øst-Europa, India, Pakistan, China)
  \item <<early adopters>> som ønsker å prøve ut en radikalt annerledes
    tjeneste (<<mid>> til <<high-end>> brukere)
  \item applikasjonsutviklere
\end{itemize}

\subsection*{Verdiløfte}

Det verdifulle vi kan tilby våre kundesegment er
%
\begin{itemize}
  \item En helt ny type mobiltelefon, hvor produkt og tjeneste er ett
  \item Sikkerhet, beskyttelse mot tyveri og tap
  \item En dynamisk tjeneste hvor man kan betale for økt lagringsplass og
    prosessorbruk
  \item Backup (alt ligger på skyen)
  \item Billig etableringspris, forskjellige abonnementsnivå og
    -priser (differensiere på lagringsplass brukt, osv.)
  \item Økt batteritid
\end{itemize}

\subsection*{Kanaler}

Vi ser for oss å selge selve enhetene i tradisjonelle butikker
(elektronikk-kjeder, mobilbutikker), men også i små kiosker (for eksempel
Narvesen, kiosker på gaten i storbyer, osv.). Dersom vi kan tilby integrerte
SIM--kort kan telefonen aktiveres på stedet\footnote{Dette er litt utenfor
oppgaveomfanget, men en svært fristende mulighet: Å kunne tilby produktet
sammen med et mobilabonnement.}.

For å øke bevissthet må vi drive massiv markedsføring. Dette bør gjøres noe
utradisjonelt, dog. For eksempel, i store, tett befolkede byer, ser vi for oss
kampanjer på gata. Med <<ready-to-go>> mobilavtale kan vi selge enhetene på
gaten. For å få startkapital og skape blæst rundt konseptet kan vi også forsøke
oss på \textit{crowdfunding} gjennom nettsteder som Kickstarter.

Kundestøtte skjer gjennom applikasjoner i selve mobilenheten, eller på nettsider.

\subsection*{Kundeforhold}

I stor grad ønsker vi å kunne tilby selvbetjening for kunder, hvor en kan gjøre
endringer på abonnement, sende inn feilmeldinger og få hjelp.

Utviklere må enkelt kunne distribuere sine applikasjoner på plattformen vår.  Det alle
de store produsentene gjør er å godkjenne hver applikasjon. Dette krever en stor
investering for å få til.  I begynnelsen kan vi tilby kun vår egne applikasjoner, før vi
åpner for eksterne utviklere.

\subsection*{Fordeler}

Ved å lansere vårt produkt får vi en <<first-mover>>-fordel \cite{bessant}.
Imidlertid er det stor sjanse for at etablerte produsenter kan kopiere vår
modell med en med hybrid løsning, hvor noen tjenester kjører i skyen. Derfor må
vi satse stort på en kompromissløs tynnklientløsning.

\subsection*{Inntektsstrømmer}

Vi ønsker å gjøre det enkelt og billig å bli kunde. Derfor må vi subsidiere
mobiltelefonen kraftig og tjene inn dette på abonnementstjenesten.

\textbf{Faste månedsbeløp --- }
Det mest grunnleggende abonnementet bør dekke et typisk månedsforbruk med data.

\textbf{Betale for bruk --- }
Samtidig må det være mulig å betale mer for krevende brukere. For eksempel, de
som bruker mye lagringsplass kan betale per brukte gigabyte.

\textbf{Andre muligheter --- }
Til slutt fins det en mulighet for å selge brukerdata. Dette er et svært
kontroversielt tema, spesielt i Europa, og kan skape et dårlig image.

Imidlertid bør en ha en mulighet for anonymisert samling av bruksmønstre. Dette
for å kunne forbedre automatiserte tjenester.

\subsection*{Nøkkelressurser}

Det vil sannsynligvis være svært viktig å sikre seg patentbeskyttelse på disse
idéene tidlig. Ikke bare for konkurransebeskyttelse, men også for å kunne
stille opp med noe spesifikt i tilfelle vi blir saksøkt av andre.

\subsection*{Nøkkelaktiviteter}

Før man kan selge den første mobiltelefonen kreves det enorme investeringer i
produktutvikling, etablering av infrastruktur for tjenere.

Parallelt med dette må man skape en effektiv infrastruktur for administrasjon,
salg, kundestøtte og lignende.

\subsection*{Nøkkelpartnere}

Vi nevner kort \textit{mobiloperatører}, \textit{komponentprodusenter} og
\textit{patentbyrå}.

\subsection*{Kostnadsstruktur}

De viktigste kostnadene i første fase blir forskning og utvikling, samt
etablering og drift av datasentre.

