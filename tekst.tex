\chapter{Forretningsidéen}

\section{Tynnklient}
Noen problemer med dagens smarttelefoner er at de er \textit{kostbare å
produsere}, de er \textit{dyre for kundene}, de har \textit{lav batteritid}
og produsentene sliter med å lage dem \textit{tynnere og lettere}. Et annet
problem er ved tyveri; data lagres ikke automatisk på nett, og kan dermed gå
tapt eller komme på avveier.

Alt dette koker ned til at alle eksisterende smarttelefoner på markedet er \textit{tykke
klienter}. En tykk klient betyr at alle oppgavene utføres av en mikroprosessor
på selve mobiltelefonen. Den trenger da lokal lagring og minne. 

Det motsatte av dette er \textit{tynnklienter} --- her er selve enheten et
relativt dumt grensesnitt mot en nettverkstjener. Tenk en PC som mottar bilde
og lyd i sin helhet fra en tjener. Alle tastetrykk og musebevegelser sendes til
tjeneren, og tjeneren er den \textit{egentlige} PC-en.

En \textit{tynnklientmobil} vil dermed være en mobiltelefon som er svært simpel
komponentmessig, men som i prinsippet kan være grensesnittet mot en kraftig
superdatamaskin, om man så ønsker det. Av dette får man umiddelbare fordeler:

Istedenfor å kjøpe nye mobiltelefoner kan man bare kjøpe raskere tjenester i
skyen. Videoredigering er enkelt, man oppskalerer jobber etter det man trenger.

\begin{itemize}
  \item All data ligger lagret på en remote server. Mobiltelefonen inneholder
  KUN en UUID og et passord i minne. Ved tyveri er mobilen VERDILØS. Faktisk så
  verdiløs at det er TOTALT unødvendig å stjele en; man MÅ ha abonnement for å
  kunne bruke den, dermed er tyveri eliminert.

  \item Det blir ingen mer oppgradering, det skjer automatisk på serveren.

  \item Ved tyveri eller tap kan man bare få en ny mobil og fortsette der
  man slapp. Alle apper, alt man jobbet med, ligger der fremdeles.

  \item Krever minimalt med komponenter: Kun AD/DA konverter, display,
  minimalt med minne, mic, høyttaler, GSM+radio, evt et kamera. Krever en del
  mer da, for den trenger jo komprimering av bilde data og sånt. Derfor driter
  vi i kamera foreløpig. Mindre komponenter betyr tynnere mobil og billigere å
  produsere.

\end{itemize}

Ulemper: Skape et økosystem, vanskelig å konkurrere med apple og samsung.etc

\section{Abonnementstjeneste}

Den ekte inntekten ligger i abonnementtjenestene. Selve mobiltelefonen bør være
subsidiert så mye at den i praksis er gratis for kunden. Ved at kunden har all
data i skyen blir det høy <<lock-in>>, og man vil da tjene penger på
abonnementet. Vi ser for oss at produktet selges med innebakt mobilabonnement,
dvs at SIM-kort allerede er installert, og man slipper å opprette avtale med
hverken Telenor eller noe som helst. Det ordner vi. Hvorfor? For å gjøre det
enkelt.

Når man ringer til telefonen, så gjør det via Asterix og SIP. Dvs tjeneren
mottar/starter opp samtale, router til mobilenhet via datatrafikk over nett.

Apper må kjøpes utover abonnementet. Abonnementet bør dekke en god del apps og
tjenester. 

\section{Fri data, over hele verden}

Som TomTom, får avtale med produsenter. Dette er for å differensiere oss fra
andre produsenter.

% - forklare idéen og konseptet
% - forklare HVORFOR det er en god idé

\chapter{Forretningsplan}

I denne skissen har vi valgt å bruke <<the business model canvas>>
\cite{oswalder} som utgangspunkt for en full forretningsmodell.

Den består av følgende punkter:

\begin{itemize}
  \item Nøkkelpartnere
  \item Nøkkelaktiviteter
  \item Nøkkelressurser
  \item Verdiforslag (<<value propositions>>)
  \item Kundeforhold
  \item Kundesegment
  \item Salgskanaler
  \item Kostnadsstruktur
  \item Inntektsstrømmer
\end{itemize}

\chapter{Utførelse}


% Datainnsamling
% - Markedsdata (markedsstørrelse, low-mid end)
% - Demografiske data, inntektsgrupper osv, spredning
% - Priser for deler og komponenter, priser for oppskalert produksjon
% - Ha et realistisk forhold til produksjonskostnader
% - Utviklingskostnader, må da se på hvilke hovedfaser som må gjøres
% Kontakte feks Telenor Group Innovasjon
% Kontakte Innovasjon Norge

\cite{bessant}.
